\documentclass{article}
\usepackage{tikz}
\usepackage{setspace}
\usepackage{amsthm,amsmath,amssymb}
\usepackage{mathrsfs}
\usepackage{geometry}
\geometry{a4paper,scale=0.75}
\usepackage{amsfonts}
\usepackage[utf8]{inputenc}
\usepackage{amssymb}
\usepackage{amsmath}
\usepackage[all]{xy}
\usepackage{amsmath,amscd}
\title{\heiti 分 \ 析 \ 哲 \ 学 \ 概 \ 论 }
\usepackage[UTF8]{ctex}
\author{\kaishu 授课教师\ 叶闯 \\\kaishu 课程笔记\ 张益豪}
\everymath{\displaystyle}
\begin{document}
\maketitle
\tableofcontents
\newpage
\renewcommand{\baselinestretch}{1.5}
\section{分析哲学的风格、历史与现状}
\subsection{分析哲学的历史定位和方法论}
首先,我们必须认识到分析哲学的历史定位是一门"新"的哲学课目。事实上,西方哲学史在十九世纪之前存在着一个基本的相对统一的主线,而分析哲学的出现实际上是来源于对认识论问题的分叉。或者说,这是解决认识论的难题的两个路向,亦即现象学传统和分析传统。这里,现象学传统的研究意向是偏向于直观和现象的认识,而分析哲学侧重于理性的概念分析。

\textbf{\kaishu 参考内容[1].} \ {\kaishu Husserl是现象学的创立者。他把现象学发展为先验唯心论,包含了方法论向本体论的过渡,《关于纯粹现象学和现象学哲学的观念》等书提出“现象学还原”和“先验自我”对世界的构造。现象学主张只有现象(感觉材料)可以被认识,同在现象背后引起现象的东西是不存在或不可知的现象论有严格区别。现象学的现象不是指同实在或本质严格区分的、仅仅通过感官才获得的经验,而是指包括感觉、回忆、想象和判断等一切认知活动的意识形式。}

分析哲学的研究方法和风格的研究方法如何?我们注意到,它在提问的方面同传统的哲学提问的方式是有异的。对于某个具体的命题,分析哲学首先要求对问题本身作一分析,考察问题本身的实际语义,而不是立即着手解答这个问题。在重构问题(使之成为某种严格标准的逻辑语言,对其进行清晰的分析)之后的一步要求我们把问题{\heiti 规整化},即把问题改写成更清楚的形式。在此之后进行的一步叫做语义上升(Semantic ascention)\footnote{这一点在现在看来是非必须的。分析哲学早期的一个口号是“一切哲学问题都是语言学问题”,它在当前的研究上被很大程度上抛弃了。},亦即把重新表述或规整化后的问题进一步转化称一个关于语句或语言的问题。此后,以逻辑的(或者至少是严格概念化的)方式陈述回答(表达为语句)的真值条件。这之后,我们就可以开始着手解答这一问题。

如何解答一个哲学问题呢?我们类似地将其分成四部分。

{\kaishu 第一步.} 陈述已有的他人的论证(一般地,利用个人理解重述他人的论证),之后指出其中的谬误。

{\kaishu 第二步.} 陈述自己的论证来表明自己提出的真值条件在某种程度上被满足了。

{\kaishu 第三步.} 阐述我们做到了什么,即评估自己的回答的成果所在。

{\kaishu 第四步.} 设想关于论证的可能的反驳,并某种程度上预先地给以回应。

\subsection{简要的分析哲学历史}
分析哲学早期面对的是认识论问题的延续。在传统的哲学研究当中,那种模糊的研究方式令这些熟稔理性和科学方法的新的哲学研究者们感到失望,他们希望利用科学的方法使得哲学能够走向为严肃的理性知识辩护的目标。

早期(十九世纪和二十世纪之交开始到二十世纪五十年代初)的方向大体分成三个路向:G.Frege的方向,他要求为数学知识寻找语言和逻辑的辩护;B.Russell的方向,它将直接的经验知识同推理的知识结合,分析其中语言表达式的根本方式上的不同刻画二者的关系;逻辑实证主义的方向,通过分析观察陈述到理论陈述的关系为科学奠基。

\textbf{\kaishu 参考内容[2].} \ {\kaishu 逻辑实证主义是借助现代逻辑对语言进行形式分析并试图建立起一个形式化的人工语言以及系统理论,以便于更好地进行科学概念和科学陈述的重新构造。它是经验主义的代表。本观点大体可概括:把哲学的任务归结为对知识进行逻辑分析,特别是对科学语言进行分析;坚持分析命题和综合命题的区分,强调通过对语言的逻辑分析以消灭形而上学;强调一切综合命题都以经验为基础,提出可证实性或可检验性和可确认性原则;主张物理语言是科学的普遍语言,试图把一切经验科学还原为物理科学,实现科学的统一。它的观点是,哲学不是一种知识的体系,而是一种活动,一种澄清或确定命题意义的活动。

它拒斥形而上学的原因是:形而上学命题割断了和经验世界的联系,在经验上、理论上,即在认识上是无意义的。所谓形而上学命题没有经验意义,它是指这类命题所涉及的对象不在感觉经验的范围之内,既不能通过经验予以证实,也不能通过经验予以否证。换一句话说,即形而上学命题不能在经验范围内确定其真假,而一个没有真假值的命题由于没有断定性的内容,因而不能给我们提供任何知识,即对增进我们的认识毫无帮助。在逻辑实证主义那里,意义一词总是从认识性的意义这个观点上来理解的。据此,形而上学命题都是一些无意义的伪命题。卡尔纳普说,我们既不肯定也不否定这些论题,我们是拒斥这整个问题。}

中期(上世纪五六十年代之后)产生了新的一个突破。实证主义在解决基本难题上获得了失败。实证主义指出,传统哲学是无意义的语言堆积,它所研究的问题是模糊不清的;这一时期的分析哲学放弃了部分的实证主义的基本原则,承认了部分的传统哲学的问题的意义而走向了在元哲学立场上坚持实用主义和自然主义相结合的一种新的(更彻底的)经验主义。这击破了实证主义的信条,为传统哲学的问题赋予了一定的合法性。这种新经验主义的代表就是奎因。进一步地,心灵哲学得到重新的兴起,它打破了传统的分析哲学问题范围的进取,乃至于克里普克(Saul Kripke)直接找回了传统的形而上学问题\footnote{这在传统分析哲学的看法上是完全没有意义的,但他把这些问题找回之后利用了新的分析的风格去研究了这些问题,实现了超越纯粹分析的突破。}。

当前,分析哲学成为了发达国家主要哲学研究的主流风格,它被广泛地接受称为讨论和解决哲学问题的方式。事实上,整个分析哲学的范畴在这一阶段就被扩充成为一种研究问题的方式(而不是某个特定的领域)。分析哲学发展出了两种研究风格,一者是利用语义学工具(逻辑的或者形式的)重构和分析所研究的问题,另一者是深入分析日常的语言,并利用这种结果说明相应的哲学问题。前者在时代的演变中走向了现代哲学文献所通用的表达和论证模式,而后者使得对日常语言的直觉的研究也要求了现代逻辑的工具和语义学的模型。
\section{语言哲学及其任务}
\subsection{基本问题和领域}
语言哲学研究的基本问题包括这样的两个方面,即对于表达具有重要哲学意义的概念的语句及其分析和关于语言表达本身的哲学讨论。前者是哲学问题的语言化,而后者是针对语言的哲学思考。这里,后者是现代分析哲学的主要研究对象。

它事实上是早期语言哲学的主要领域(在概念上也是最为基础的领域,具备奠基性),语言哲学的结果和分析范式是分析哲学的基本的解决问题范式的工具;在非一般意义上的情形其与认识论和形而上学、心灵哲学等具有密切的联系,它把上述领域所涉及的问题严格化了。

\subsection{语言哲学问题的历史回顾}
这部分主要分为三条线索。

第一条线索是问题的线索,当具有重大哲学意义的命题被利用语言哲学的方法严格化了之后,语言哲学的问题就转向了对语言这一客体本身的分析;随着语言哲学研究的演进,解决语言表达及其意义的一般性的讨论上升成为了主要问题。

第二条线索是研究风格的区别,这里形成了两个路向,即形式语义学研究纲领及语言使用分析为导向的研究纲领。后者的特殊性在于它考虑到语言的实际使用,亦即语言的心理要素、社会要素、语境要素等。

第三条线索是针对名字的指称问题的对立的主张,包含直接指称理论和指称的描述理论。

\subsection{指称问题}
指称问题是一个最基本的问题,它包含这样的一些问题。

{\heiti 一般性问题.}对某对象的指称是一种什么关系?

什么决定了某一问题所实际指称的对象?

什么情况下将发生指称失败的情况?

什么时候将发生指称的转移和改变?它在什么时候以何种手段发生?

指称的直接性和严格性的关系如何?

名字是否可以一般地被解释成谓词\footnote{用来代替或者展示其客体性质、特征或者客体之间关系的词项。个体词是可以独立存在的事或物,包括现实物、精神物和精神事三种。谓词则是用来刻画个体词的性质的词,即刻画事和物之间的某种关系表现的词。这个命题旨在分析一个指称(或者个体词的表述)能否被认为是一种谓词和关系。}?

{\heiti 特殊问题.}自然类词指称什么,其指称是如何被决定的(若它存在)?

复合的表达式的指称如何?确定的表述是否有指称?

{\heiti 相关的形而上学问题和其他问题.}是否存在一种情形使得一个词成为指称词,但它指称不存在的对象甚至不指称任何对象?指称是否依赖于语言表达主体的认知把握?

{\heiti 语言装置与真理/语言装置与本体论的承诺\footnote{科学理论中的每一个概念,虽然它本身是人类语言中的一个符号,但功能在于指称语言之外客观世界中的一个对象。当人们对“我们是否认为这个对象是存在的”这一问题作肯定回答就实现了指称的本体论承诺。认为在语言所指对象具有不可观察性时,这种承诺尤为必要。这是人们继续谈论话语或研究理论的前提。从科学实在论的发展趋势看,人们对指称的本体论承诺是在不断缩减。}.} \ 对于语句的本体论承诺问题,事实上包括了“断定的叙述一个语句是否承诺量词约束的对象”这一问题。

{\heiti 语境对于语句的语义解释的作用\footnote{事实上对一些有争议的语句,语境的作用是影响语句的真值条件、真值评价,产生了语境主义和相对主义的争论和语境分析的问题。}.}指称是否是唯一的使语境起作用的可能的方式?

{\heiti 具备特殊哲学意义的语句或语句类的语义学解释.}反事实条件句\footnote{例如事实上A与B是同时发生的,其共同作用是C,那么“若A在B之前发生则不会产生C”就是这样的语句。}、否定存在语句\footnote{例如“不存在道德事实”。}、命题态度语句\footnote{除了提出一个命题之外,还包含命题态度,即说话者对这命题的断定,表达他对这命题抱有信念。例如"我希望A成为B"。}、具有特殊事态的语句等的语义学与逻辑以及模糊性问题。

{\heiti 由处理复杂语言现象引发的新语义学理论.}二维语义学和使真者语义学\footnote{为了分析指称真值条件,其使真者语句本身的真值问题和语句分析产生的递归问题。}等。

{\heiti 具有哲学背景的具体问题.}引用指称\footnote{引语的指称作用由何种符号决定,这一预言成分以何种意义成为指称,指称什么等这样的三个子问题。}所引发的问题等。
\subsection{语言哲学与语言学}
语言哲学同语言学还具备一定的距离。其在外观上是相当相似的,其工作的内容和工具之类是相近的,在这种情形下唯一的不同是希望解决的问题的不同。需要指出的是,二者的哲学兴趣事实上是不同的。语言学家将在对某些具体问题(例如,命题是抽象的实体还是心灵的实体)上的立场保持中立,但哲学家必须深入探讨这一问题;另一方面,二者又在概念上可能(在实际上经常)具有相同的研究方向。
\section{指称理论:它的基本争论与问题}
指称理论是最基本的语言哲学内容。可以认为,指称是语言同世界发生联系的直接通道。指称理论试图解决的一般性问题是,一类名字的指代对应到何种现实对象(语义学);一类名字所指代的对象由什么决定(元理论的基础性问题。可能是概念、心理、因果或其他外在因素决定的)。指称理论的元理论问题解决指称理论的基础建构。例如,“$X$”类表达式是否是一个指称性的表达式?进一步地,指称性的标准是什么,如何指定一类词(例如特定的谓词亦或语句)或者其特定用法是“指称的”?这里的直接动因是,“弗雷格之谜”的提出。

\textbf{\kaishu 参考内容[3].} \ {\kaishu 弗雷格在其影响深远的名著《论含义和指称》一文中,弗雷格提出了下列这个对所有意义理论的挑战:给定两个专名“$a$”和“$b$”,如果“$a=b$”是真的,那么“$a=b$”和“$a=a$”怎么会在“认知价值”(cognitive value)方面或者说是在认知信息内容方面有所不同呢?很明显地它们在这方面确实是不同的,因为第一个陈述是后天才可知的因此是能给我们新信息的,相反第二个陈述是先天就可知的因此是不能给我们新信息的。然而,假设“$a=b$”表达了“$a$”的指称和“$b$”的指称的同一性关系,并且“$a=a$”表达了“$a$”的指称和它自己的同一性关系;那么如果“$a=b$”是真的,它表达的就是和“$a=a$”所表达的完全相同的同一性关系。}

这个“弗雷格之谜”的解释方式具备两种解决方案,称之为“符号方案”和“对象方案”。

就符号方案而言,若认为这两个词汇具备着符号上的关系,那么首先应当认识到符号的任意性和抽象性。通过符号很难获得同实际对象等同地具备认知意义的结论。第二,直觉上其符号的陈述针对着同一个对象;再者,使用(use)和提及(mention to)在符号的等同下将这个区别混淆了;另外假如认可$a$和$b$只具备符号上的区别却不具备实际上的区别,就必须解释$a=b$问题的真性需要利用实际的经验来发现。

就对象方案而言,假定认为$a=b$是对象关系,那么两个表达的相同将舍弃任何认知区别;这样,对对象到自身的同一性将难以解释,而认知的差别可以通过符号的说明来解释,却很难通过对象的区别来阐释,因为仅仅对对象的同一性而言,符号本身的差别就被抹杀,进一步地,这两个表达式就成了分析上为真的一类命题\footnote{按照Frege在“Sense and Reference”中的理解,针对“a=a”与“a=b”中的“a”与“b”,如果只有外延对象作为其语义值的话,当“a=b”是真语句的时候,“a=a”与“a=b”都是在表达对象的自我同一的内容,二者就没有任何实质意义上的语义值的区分;而Frege时期相关术语并不是十分标准化,按照当代学者的解读,Frege的上述刻画并不是严格语义层面的,只是反映出在上述设定中,“a=a”与“a=b”在认知上是无差别的,都反映了某种先天可知的法则性真理(任何一个对象都是自我同一的)。在Frege撰写“Sense and Reference”时期,Frege除了对上述诉诸对象解读方案不满意之外,他对于其“概念文字”时期的符号处理方案也是不满意的(大致是指,“a=a”与“a=b”的差异仅仅是因为后一语句出现了“a”与“b”两个记号来标定同一对象的话,“a=a”与“a=b”的认知差异即使存在,也是十分琐屑的(trivial))。}。

这样,弗雷格给出了一种解决方案,将{\heiti 指称}同{\heiti 意义}做了一个划分。弗雷格的解决方案认为,一个{\heiti 名字}(name)具备两个语义学要素,即指称和意义。一个名字具备其本身的意义的同时,指称一个自己的对象。意义是指称被给出的方式,并且指称是由意义刻画的。事实上,从意义到指称在这套体系下是一个多对一的关系\footnote{我们不妨这样解释:考虑一个实际对象的两个名字“$A$”和“$B$”。$A$和$B$蕴含了其所指代的对象本身(亦即指称),但是它们又从不同的意义角度决定了这个指称的给出方式。给定这个实际对象$C$,那么我们认为多对一关系就是把两个名字$A,B$改写成两个二元组$(A,C),(B,C)$,这左元到右元的对应是存在的,但多个意义可以对应同一个指称。}。
\subsection{弗雷格框架的基本内容}
根据这种方案衍生的思想体系叫做{\heiti 弗雷格主义}。在弗雷格的框架下有若干的基本性的原则:首先是{\heiti 反心理主义\footnote{心理主义:把社会现象归结为心理现象的思想。心理主义具有个人主义和唯名论的倾向,只有那些用个人的心理品质解释社会事实,同时不考虑这些个体的相互作用问题的观点才被认为是这里所讨论的心理主义。}原则}。这一原则认为对同一个语言表达其意义是固定的,而表述者在述说的过程当中固然也可以有多种心理状态,但后者不在这一体系下被考虑;在另一个论证上,我们可以说词意是我们所考虑的对象,而不可能指代在我们在考虑这些对象的过程中形成的观念。第二个原则则是说明,一个语句的真值条件就是它的意义本身。或者说,语言表达的意义就是它对自己出现于其中的那个语句的真值条件(事实上就是意义)做出的贡献。第三个原则是{\heiti 组合性原则和语境原则}。这里,一个语句的意义或者指称仅仅由其组分的意义或指称决定\footnote{这就是组合性原则。比喻地讲,一面墙由若干砖块及其堆叠方式构成,那么它就由这些组分完全确定,而不依赖什么其他的东西。}(也包含组分在语句中的安排方式)。语境原则则是说作为组分的语词的意义完全在语句本身的语境当中,它的内容不会超出它对所在语句的意义能做的贡献。
\subsection{罗素的摹状词理论}
产生新理论的动因是罗素所面临的三个问题。

{\kaishu 空名问题.} \ 在一个语句当中会出现破坏排中律等\footnote{排中律是说,$A$和非$A$这两个命题当中必然有且仅有一个成立。但是在空名的情形下将自然地产生疑问:若$A$所指代的对象完全不存在,我们怎么说明其中的某一个是正确的呢?这就打破了基本的排中律。}的问题。例如某个语句中存在这这样的一个指称,它所指代的对象并不在实际的世界当中存在,那么这个语句的真值条件将无法阐释,引起了语义和逻辑的困难。

{\kaishu 命题态度语境下的同指替换.} \ 罗素在1905年提出了一个例子\footnote{罗素说,George IV想知道Scott是否是Waverley的作者,而事实上Waverley正是Scott写的。所以,我们可以将Scott和“Waverley的作者”取等号或者相互替换,那么问题就变成了“George IV想知道Scott是否是Scott”。但是,用罗素的话来说,“人们并不认为欧洲的这位头等显贵对同一律感兴趣”。},说明了这种同指替换的问题。

{\kaishu 否定存在难题.} \ 我们给出一个语句,其否定了某个主词的存在。那么问题就是,为什么一个语句会被判定为真的同时其的主词所指代的对象根本不存在\footnote{在日常语言当中,我们总是默认研究围绕一个主词的命题是否为真的前提是主词所指代的对象具有某种性质才为真。但是否定存在命题否定了主词这个对象本身的存在性,使得其听起来好像是矛盾的——但是它在直觉上却不是错的。}?

Russell提出了{\heiti 摹状词理论}来解决这一问题。这一理论摒弃了弗雷格的复杂的涵义概念,提出了一种叫做{\heiti 摹状词}的对象。它不是传统意义上的一个名字,是一个“不完全的符号”。一个摹状词只能是语法主语,而不能够成为逻辑主语\footnote{这里的逻辑形式是,$n$是逻辑专名,那么$n$是$F$的逻辑形式可以写成$F(n).$};我们否定了摹状词的指称性。因而摹状词不能够作为独立的语义项使用,它不能单独地具备意义,而是只能在适当的语句里起作用,换言之,一个摹状词具有着语境意义。任何一个摹状词在逻辑形式正确的语句当中都不是一个独立的语法成分(自然而然地,更不可能是语句的主语)。从而,我们统一地规定:日常专名也不是什么真正的名字,而是伪装的摹状词。它的逻辑形式实质上等同于一个摹状词;这样我们就知道,事实上我们在理解问题的过程当中产生困难的原因只不过是我们忽略了一个语词中暗含的逻辑形式。

\textbf{\kaishu 参考内容[4].} \ {\kaishu Russell是如何通过摹状词理论解决他的问题的呢?Russell告诉我们,摹状词描述满足某种条件的,存在一个并且仅仅存在一个的(即数学中的唯一的)那种事物时所使用的逻辑。例如一个典型的“空名问题”可以做这样的拆解:

“当今的法国国王($X$)是秃头($F$)”这句话应表述为:

\textbf{1}:至少有一个对象 $X$ 是当今的法国国王;[存在]

\textbf{2}:至多有一个对象 $X$ 是当今的法国国王;[仅存在1个]

\textbf{3}:$X$ 是秃头。[$F$($X$)]

前2个命题可以得出结论:存在且仅存在1个当今的法国国王,但是当今的法国国王并不存在,因此根据命题的合取规则,可知此结论为假命题,所以排中律没有失效。

否定存在命题解决如下:
“金山不存在"这个命题拒绝给金山赋予实在性,该命题可以表述为"有一个$X$,$X$是金的,且$X$是一座山,那么$X$就是不存在的"。

同指替换问题解决如下:
"北京是中国的首都"可以改为:有一个实体C,使得如果$X$是C,"$X$是中国的首都"这个陈述是真的,否则它是假的;而且C是北京。
}

这一理论背后的认识论动机是去解释我们基本认识模式上的区别在语言的逻辑形式上的表示中的体现。他提出直接接触所知同通过指称所知是认识(Acquaintance)和知识(Knowledge)的根本区别所在,前者(北京)可以被作为逻辑专名使用,后者(金山,当今的法国国王)则必须保留它通过指称获得的性质。

与此同时,罗素的方案尽管解决了一些问题,但是它本身还是有许多问题。首先我们无法给出罗素这种规定方式的理论依据。根据直觉,我们认为日常的名字这一对象最重要的性质或者功能是指称,但是罗素否定了名字的指称性(与此相反,弗雷格只是为名字赋予了意义来保留指称,空名则是任意的指称)。P.F.Strawson批判了这一理论,提出了一种新的解决问题的模式。

\subsection{\textbf{*}解决罗素问题的其他路径}

Strawson区分了语句(或者其他语言表达)的说出(Utterance)和使用(Use)\footnote{这是Strawson的一个最基本的区分,使得对语言的分析多了一个(或者说两个)维度。这使得一些原来似乎是逻辑或者语义学的困难有了新的意义.}。语言表达的语义学内容、有意义性与在特定场合下说话者的语句所断定和指称的对象也有一个区分\footnote{在上述两个区分之下,指称从属于了对语言使用的描述。它只有在被使用的时候才存在,语句本身没有真假,它只有在被使用的时候才可能做出真或者假的断定。按照斯特劳森的看法,没有指称的“当今的法国国王”应该是无真无假,而不是有意义的假命题:罗素在分析包含有非真实的事物的命题时,通过把指示这种事物的符号看作是摹状词,消除了这种事物存在的矛盾。它在句子中并没有指谓真实存在的事物。罗素的这种做法混淆了指称某个实体和断定某个实体的存在。我们在指称实体时只是假定了它的存在,旦并没有断定它的存在,我们也不能从关于这个实体的论述中推断出它的存在。};一个语句所蕴含的与在特定的场合下一个语句被用于一个断定时所必须假设的条件(Presupposition)同时也有一个区分\footnote{在上述三个区分之下,空名问题变成了语言使用的一个现象。其时,假设的条件未被满足,从而相应的断定便无所谓真或者假。直接断定一个命题,忽略了假设存在和空名指称的存乎被使用前提下的关系。命题的真假和意义并不相通,我们不能根据命题有无意义而严格地断定它的真假(罗素则认为有意义才有真假),因为在许多情况下, 一个句子在被使用时根本不存在真假问题。例如,“当今法国国王是秃顶”这句话无疑余是有意义的,但当它被用于指称现在的某个法国国王时却既不是真的也不是假的,因为现在法国根本没有国王。}。

另一方面,K.Donnellan提出了摹状词的两种用法。它把摹状词的使用区分为了归属(Attributive)的使用与指称的使用的划分。在归属的用法当中,如果存在一个独一无二的满足摹状的对象,那么摹状词就指称这个对象;否则它就不指称任何对象,对应地就不存在一个真值。指称用法只是在指代一个确定的对象时起作用,尽管其所指称的对象未必符合某个真值条件。当某人确实地使用了某个摹状词说话时,其心中存在确定的对象(尽管对罗素来说,他不承认摹状词的指称性)\footnote{我们还可以借助denoting和referring两种不同的表征关系来理解关于限定性摹状词的归属性用法与指称性用法的区别。按照Keith Donnellan的理解,当限定性摹状词被归属性使用的时候,该限定性摹状词就是按照类似罗素处理限定性摹状词的方法,以描述内容为标准来筛选出满足相关描述性质(在语境凸显意义上唯一的)对象的,此时,限定性摹状词与其外延对象之间的关系就是denoting的;而在指称性用法的情形中,说话者的referring意向占据主导地位,而相关限定性摹状词只起辅助性的的标记(甚至可以完全不起任何筛选)作用,这种情况下,当限定性摹状词被指示地使用时,其语义学功能大体类似于专名,而其外延对象是允许出现完全不满足限定性摹状词中的描述内容的。}。

\subsection{指称的描述理论及其特点}
指称的描述理论的主要特点是,它并不把指示的对象(而是,某种意义上,描述)作为决定指称的标准。某一个专名同限定摹状词的语义关联\footnote{在日常用语中,通名,是指对一系列相似概念的称呼,如“轮船”;专名,是指对特定概念的称呼,如“泰坦尼克号”;限定摹状词,是指用特定的谓语对概念的限定,如1912年4月沉没的那艘轮船。在这里,后两者通过某种意义统一起来了。}在特定的解释下存在统一的语义学;“思想”同指称也存在确定的关联(换言之,指称通过心理或意义对说话者透明,它与说话者具备这种层次上的联系);并存在意义对指称关系的单向性\footnote{在描述论那里,意义唯一确定地决定着某一语词的指称,它的涵义将最后决出它的指称,而不能够通过对某个对象的指称来反过来获得某个语词的意义。}。

这一理论本身同时也具备诸多的问题。几个最基本的问题是,若规定说话者的意义是“弗雷格式的”意义,那么这要求说话者具备充分的语言知识;并且,通过掌握某一语言,我们将了解意义到实际指称的决定关系,从而能直接了解很多信息,然而这其实并不符合实际的直觉\footnote{假如我们掌握了某一个语言,也就必须掌握这个语言的全部意义到指称对象的关系,而事实上我们并不需要在了解某个语言的同时知道全部的这些关系;我们讨论A的时候自然也不必要了解其的一切认识决定,比如我们提及“甲”这个人时,完全不需要了解甲蕴含的属性,但是描述论认为掌握语言先天地必须了解这种联系。}。另外,“空名”的产生也是反直觉的:若一个名字通过一个意义决定一个指称的唯一对象,那么这个指称并不一定会存在唯一的满足“描述”的对象\footnote{例如,我们说明某个名字蕴含若干意义,每个意义是决定指称对象的一个条件(例如,这个对象是通过$A,B,C$三个条件保证的)。但是这些条件所对应的对象只有一个吗?这就未必然,从而产生了空名,亦即指称不清晰的问题。},这样将产生同实际认识相悖的“空名”。

\section{直接指称论}
直接指称论这一论述对传统描述理论提出了更进一步的反驳,主要的反驳有如下三个。

{\kaishu 模态\footnote{模态是和某种实然相区分的,它提出一种可能性或者必然性,而不是实然性。可能性的判真涉及到可能世界的叙述,即至少在一个可能世界中为真”的命题(例如“本笔记的笔者在2022年当选为美国总统”),我们在注释26那里会提到这个概念。}论证.}  假定描述论真,那么若某指称是由某个意义决定的,那么这个意义到这个指称的决定关系必然是真的\footnote{例如,根据描述论,亚里士多德具备“是柏拉图的学生”这个意义,故而“亚里士多德是柏拉图的学生”先验为真;然而不仅语句不一定真,亚里士多德也根本不必然。这就走向了模态论证。在直接指称论那里,这种必然描绘“各种可能的世界上成立”;而可能世界则是一个相当技术性的概念。可能世界的概念被用来表达模态断言,那些使用可能世界概念的人认为“实际”世界是很多可能世界中的一个。对于世界可以是的每个不同方式,都被称为一个独特的可能世界;实际世界是我们事实上住在的世界。命题的模态状态被按照“在其中它为真的世界”的方式来理解。在哲学中,术语“模态”覆盖了如“可能性”、“必然性”和“偶然性”这种观念。可能世界的想法最普遍的归功于Leibniz,他称可能世界为神头脑中的想法。更深入的讨论将在形而上学的部分继续说明。}。但是这个论证未必是必然的。

{\kaishu 认识论论证.}  如果描述论真,我们只需要通过我们对名字的意义的掌握就知道一个意义决定的是什么指称;但是我们未必能够仅仅通过对某个名字的认识就知道这个名字蕴含的全体性质\footnote{例如,按照描述论,我们应该仅仅依靠对名字的意义的掌握知道亚里士多德是柏拉图的学生。然而,这个知识并不蕴含于“亚里士多德”这个名字当中,尽管我们没有通过学习知道这个知识,亚里士多德还是可以成为一个合法语素。与此同时,还存在亚里士多德不是柏拉图的学生的可能性,但描述论彻底否定了这一点,从而导出了不符合我们直觉的结果。}。

{\kaishu 语义学论证.} 
假如某个名字的意义为全体人类所公认,但是这个“意义”被推翻了之后,这个名字将指称新的意义所决定的对象;然而,前面的那个指称在我们的直觉下,不会因为意义的改变而改变\footnote{一个例子是:Thales的意义是“那个认为世界由水组成的哲学家”。然而如果那个作者搞错了,历史上真实的叫做Thales的那个人根本不是什么哲学家,这个观点归属于一个我们并不知道的隐士哲学家。那么,Thales在描述论的意义下只能归属于第二个哲学家了;但是,根据大众的普遍直觉,第一个人才应该被叫做Thales。}。

\subsection{直接指称理论的基本论点}
直接指称理论是这样解决上面的两个问题的。首先,专名在如下意义上是{\heiti 直接指称}的:首先名字指称对象,而不由任何联想到名字的描述去决定;一个名字指称它{\heiti 实际上}指称的那个对象,是一种语义学的规定。其次,一个专名背后不蕴含任何的描述性内容,其的唯一的语义学功能就只是它指称的对象,并且在全体可能的世界当中都指称着同一个对象。也就是说,指称是先于描述存在的,我们根据名字本身来获得一个对象,一个名字就是一个指称;这个决定关系在被规定的意义\footnote{即是说,这个性质是根据某种规定而不是满足了某个特定的性质而体现严格性的。}下是严格的。

进一步地,我们说一个名字指称它实际上所指称的对象的依据是一个基础性事实,存在着一个因果的链条,最终从名字的使用者连接到那个对象\footnote{直接指称语义学具备一个形而上学的基础,即指称因果论。有论者认为,指称的因果理论事实上并不构成一个严格的哲学理论, 它至多是一个图景, 这个图景我们可以很容易地用指称的初次确定 (“命名仪式”) 加上指称的传递 (“因果-历史链条”) 来概括。在典型情况下, 名称N被说话者S用于指涉对象O, 如果S对于N的使用与O的初始命名仪式具有因果关联。}
。对于自然类词而言,我们这里有类似的直接指称的机制;这使得这里的论证不是语义学的论证,而是外在(或者说脱离语义体系)的论证了\footnote{有论者称,直接指称论的语义学概念导出的哲学结果(认识论)同形而上学有严格的意义区分,两类命题具备本质不同的为真的条件。例如,我们在形而上学的意义上指定水是$\mathrm{H_2O}$,那么它就必然是$\mathrm{H_2O}$;而直接指称论允许认为水的化学式不是$\mathrm{H_2O}$,但是从外部意义上一致。还有论者认为直接指称论蕴含着本质主义。因为直接指称理论认为专名是严格的指示词,使得本质主义可以理解;这一论点的支柱是一个直觉,即我们对一个不加描述的东西(例如我说“CITP”,我们不知道这个词汇的任何意义,但是却可以直接使用它)述说其在各种可能意义的情形。关于陈述的真的必然性断定同对对象本质性质的断言之间关系如下:N是O对象的指示词,F是表达性质P的谓词,那么P是O的本质性质这一断言等价于$\exists \mathrm{N}\Rightarrow\mathrm{F(N)} $。有人进一步地认为,这直接指称论的语义学蕴含或引出这样的结果:存在着先天的偶然命题和必然的后天命题。例如,“I am here”,它先天为真同时具备偶然性,因为我们无需做出任何分析就知道它是对的,而不需要调查每一个“我”的位置是否是“这里”;再例如“水是$\mathrm{H_2O}$”这命题是后天发现的必然性。这里,一般认为必然性刻画了一种无论何种条件下都会发生的性质;先天则刻画了无论何种条件下都成立的某种认知条件。事实上,一般可以通过真值条件的来源来判断这里的先后天等的区别(有偏于综合的事实判据和偏于分析的意义判据的两种解释)。这两个条件听起来具有某种统一性,而事实上它在实证主义者那里就是一致的。}。
\subsection{直接指称理论的贡献}
直接指称理论有这样的贡献。首先,它解决了上述三个论证当中的反直觉的结果,并方便了某种关于意义的外在主义理论和关于语境敏感词的二维语义(因有必然性的两种意义,认识论的与形而上学的);它以自然的姿态引入了一个讨论形而上学的本质问题、模态问题等传统问题的被普遍认可的框架,乃至于使得科学进步有意义\footnote{在描述论的框架下,希腊人认为“原子是不可分割的最小部分”;而这样下来,它就框定了这个科学概念而不必向下发展了。但是我们现在知道的原子当然是可分割的。这就实现了某种意义上“为科学辩护”的目标。}。
\subsection{指称描述理论的自我辩护和直接指称论的回击}
指称描述理论提出了一个自救的办法,其技术性手段是严格化算子“Actually”\footnote{这是试图解决意义不能唯一决定指称的手段。它使得一个名字的若干意义确定地对应到一个指称,以期解决注释24那里的问题。};同时将直接指称的指称传递因果论链条包络到名字的意义内部,除此之外要求元语言要素来给定一个意义(例如确定谓词“被叫做”),诉诸语言使用解释等\footnote{但是,在广泛的认识上,指称描述理论的这种辩护是相当无力的。尽管可以利用Actually算子去严格化某个名字,但是仍然不能解决语义学论证提出的意义到指称的变更的过程。}。

直接指称论进一步地反驳了这个论调。关于Actually算子,它表示这个方案并不能在“所有可能的世界下通用”;不同的“世界”的同胞之间可以由对某些个体的信念实现共享和公认,但是我们没有办法实际相信在我们这个实际的世界有些什么。它不仅没有正面回应认识论论证问题和语义学论证问题,且没有办法使得一些在非实际性情形下才成立的蕴含关系不再成立。一般地,“若X偶然的是F,那么可能在某个W下不是F”在一般的解释下都成立,但是这个实际性算子并不能使得“X在某个F实际地不是F”在单一的解释下不能成立\footnote{例如我们提出“所有的天鹅都是黑的”,那么这个实际性算子在发现了白天鹅之后只能把这些天鹅称之为“天鹅*”,因为在单一解释模型下那种天鹅的形象从定义的意义上就是黑的,它定义了“天鹅是黑色的”。}。

\subsection{在两个方案之间的解决方案}
到这里,我们注意到描述论和直接指称论似乎都有着不和谐的部分,从而出现了若干的解决方案,试图形成一种综合二者的解决方案;一种解决方案是通过认识论倾向的二维语义学方案,这一方案我们会在后面更深入地说明;另一种解决方案是名字的谓词理论。在直接指称论那里,一个名字就获得一个指称,但是在一个名字的专名性质\footnote{例如,Croatia州居住着若干个Nikolas。在专名的意义上,这句话同Croatia州居住了若干Dog没什么区别,而后者是一个自然类词,所以显得符合语言的直觉。}上没有得到很好的解释,而这一谓词理论试图解决这一推广的问题。一种较弱解释是,专名既可以用作类词,也可以用作固有意义的专名,在适当的环境下有不同的用法;但更强的解释认为,全体名字事实上就是谓词。谓词理论的拥护者D.G.Fara这样解释专名的类词性的矛盾:他指出全体的名字全都是谓词(而不是某些情形下);当我们要使用某个词汇的专名性\footnote{既然若干Nikolas都存在,那么就统一地认为Nikolas和Dog没有区别,而我们使用Nikolas is running这个句子的时候已经默认了一个
专名的指称的存在。},我们认为这个名词的谓词前具有隐藏的定冠词,使得“John”以“(The) John”的形态同“The dog”统一起来,在这个谓词的外延范畴内寻找一个合适的对象成为指称。这理论一定程度上协调了类词属性的问题,但是它不能解决前述在直接指称论那里“无任何描述(意义)的个体”也可以被使用的那种直觉的意义问题。

后续的一些研究包括语用学的描述主义和认知的(心理的)因果理论。这里我们主要讨论后者,描述主义在后面将会再解释。心理的因果理论同样的取定了描述论和直接指称论的中间道路,它承认名字的指称性质;但是一个名字在被使用的过程中具体到达指称的方式不仅是那种认识上的因果论,也可能是心理上的因果论。它提出了一个“心的文件”(Recanati's Mental Files)理论\footnote{这理论指出了某种描述性的内容,即一个名字的指称在心灵当中有一个“心的文件”来存放,它是决定某个意义到指称的那个方法。然而,这和描述论有根本区别;这理论的那些描述不来自于某种先验的知识,而是来源于个体心灵的“心之文件”,这使得尽管获得的意义内容是错的,也可能绑定到某一个对象,这既不同于Kripke的历史因果理论和命名仪式,也不同于Frege的描述理论。}来解决Frege的同一性难题。

在2018年,一种新的理论(Frege
-Mill mixed theory)得到了提出。其的核心论点在于,它同样承认名字的核心属性是指称及其严格指示词性质,它的贡献在于言语行为内容,也承认名字并不是归给它的描述的集合;但Frege的描述也起到作用,方式是利用描述论的方式在Kripke的意义下去作为确定指称的语境依赖的前设(而不是Kripke的历史因果链条方法)。

\section{语义外在论与内在论}
所谓语义的外在论和内在论,是指语义是否依赖于一个个体的内在性质。\footnote{外在主义的一种理解是,即我们语词或者思维着的内容的含义取决于它和外部世界的因果关系之中。一句话说就是“意义不存在于大脑里,意义外在于我们”。内在和外在与某属性是否是本质属性是相异的;它只是说明这类含义是否全部取决于内在的属性。若进一步解释,一种外在属性像是个体与外在的关系属性,例如“我与某物的位置关系”;这并不意味着这种关系是本质的。}内在主义认为一个个体的语义全部地依赖于个体的内在性质,外在主义则认为其或多或少地依赖于外在的性质。故而,也可以认为是某种程度的“个体主义”和“反个体主义”。然而,关于“个体”和“内在(外在)性质”这些关键词也需要得到厘清,这就需要进一步的解释。
内在主义认为,一个“心的性质”(换言之,心理性质)是一个个体的内容性质。我们在这里指出,一个特性是“内在的”,当且仅当这个个体的性质不依赖于任何具备它的偶然对象之外的任何东西\footnote{简而言之,我们不需要在考虑一个内在性质的时候设想存在着我们之外的任何东西就可以完全考虑它;即是说它不依赖于任何我们之外的存在或假定。}(或者说它仅依赖于个体内部)。外在主义指出并非全部的这种性质依赖于内部的性质。

我们举出两个具有重大影响的支持语义外在主义的思想实验。

{\kaishu 孪生地球思想实验.}
假定存在基本上完全相同的两个世界,其一的水叫做$\mathrm{H_2O}$,另一世界的水则认为是XYZ;但这两种水给人的心灵意义完全相同(亦即人的感觉上是两种完全一致的物质)。这时,若指出外在论的正确性,我们说$\mathrm{H_2O}$时我们认为它指代水;而它们称之为XYZ。这一心理实验指出,一个名字指称什么,在我们个体的心灵状态在这个意义下没有任何区别的情况下也能够发生不同,这就说明至少存在一个外在的部分使得水的指称变化了。这是一个“自然决定”的例子。

{\kaishu 大腿上的关节炎思想实验.}
假定两个语言共同体。我们认为“关节炎”是一个关节上的疾病,而在另一个反事实共同体里,“关节炎”指代任何四肢上的疾病。这时,若某人称“我的大腿得了关节炎”,我们的共同体将认为这不可能是对的,而反事实共同体将可能承认这语句为真。注意,这个人如果在说这句话的时候是完全相同的内在状态,同一句话的语义产生区别,这仅仅依赖于两个社会共同体的公认语言事实不同。这是一个“社会决定”的例子。

内在主义论证随即提出了一种自我辩护的论证\footnote{有人认为,内在论不能同直接指称论相统一,因为后者要求依赖于外在的那种历史因果链条。}。内在主义的一种论证认为,外在主义不能解释自我知识(Self-knowledge)的存在\footnote{当然,这里反驳的是那种较强的外在主义,它认为一切心灵状态(而不是有些)存在着一定程度的非内在因素的影响。}。自我知识,即对自我的某种认知,很明显地是一种心灵状态;然而这种心灵状态却根本不取决于任何外在于个体的认识。当一个个体提出一个关于这个体本身的某种性质时,这个解释的权威性既不来源于外界也不来源于某种自然,而是这个个体本身及其内在因素。另一个问题是外在论不能够保障交流的成功。Pollock举出了一个例子\footnote{Pollock说,若某人了解葡萄的语义,却由于个人经验认为葡萄只有绿色一种颜色,那么当他向另一个人要求葡萄的时候,他所要求的葡萄在心理上是绿色的。然而另一个人若根据通常理解取来了红色的葡萄,那么这里就发生了交流的失败;这里,这个交流失败的根本原因是理解的失败,而对某物的理解根本上是内在的。所以,严格的外在论不能够完全支持一个交流的成立。},它表明交流者尽管共享着若干对于语词的语义规则,但是对所指代的对象的特征却缺乏理解;外在论,作为一种模糊的理论,被认为是不能够充分地支持这个交流过程的。

\textbf{\kaishu 参考内容[5].}
{\kaishu 这里,我们可能会对“Mental Property”产生疑问。之前我们所提及的都是语义的外部性或内部性,为何这里却出现了具有如此的心灵哲学色彩的词语呢?这里,有一种观点认为,语言哲学中的外部论 (Externalism) 我们一般叫语义外部论 (Semantic Externalism),它主张的是语义是由某些外在于说话者的因素决定的;而心灵哲学中的外部论我们一般叫 (Mental) Content Externalism,主张的是某些心智状态 ((Representational) Mental States) 的内容是由个体的外在因素决定的。因此,这二者从概念上似乎不是一回事。不过我们之所以会觉得它们好像只是强调方面不同实际上是很自然的。这可能是因为,这两种主张经常基于相同的思想实验(经由微小区别的 Interpretations )论证支持。最著名的就是 Putnam 的孪生地球思想实验;且二者在逻辑上的过渡同样是非常自然的。这根本上是基于心理机制和语义机制的真值条件的某种关系。正因为以上两个理由,即使是学术界,在几乎所有情况下都不会严格区分二者
,因此在阅读过程中更无法区分二者。然而这里的区分也是存在的,根本上是由我们是否承认上述的逻辑过渡是否自然决定的\footnote{认为二者的过渡是自然的实际上基于这样两个预设:语言是表征性的 (Representational),其语义 (Semantic) 正是其表征的内容 (Content);语言的意义的获得是由说话者的某些心智状态的内容解释的 (二者相关但不必同一)。比如经典的 Gricean Program 或者 Chomskian Program 都支持这一主张。当然,这两个主张都有反对意见,而不接受它们就可以不接受 Semantic 和 Content Externalism 的同一性,比如说 Semantic Externalism 里有一类叫 Social Externalism,主张语义是由说话者所在的社会群体决定的,这种主张不需要支持 Content Externalism。}。}
\section{意义和交流语境:语境敏感性}
这一问题同外在论和内在论的争锋遥相呼应:它提出语境\footnote{语境代表某种语言之外的因素,例如说话者、可能世界等变元。举例来讲,“Yesterday”就是一个典型的语境敏感词,它同不同的时间的指称对应不同。}对于语言的敏感性\footnote{在可能性世界中,内涵是可以从外延推导出来的。这是因为,假设一个内涵X,在可能性世界W中,确定了一个外延Y。此时,可能性世界就是自变量,Y就是因变量,函项就是X。此时,如果两个专名的外延在所有可能性世界中相同,就能倒推出来其内涵也是相同的。}。这里的核心问题\footnote{自然而然地,也存在着其他的问题:例如,我们如何界定一个知识是语义学的,抑或是较为平凡的“百科全书”式的内容?}是下面两个:

{\kaishu 问题一.}
语境因素是如何影响语言的意义或真值条件的?

{\kaishu 问题二.}
语义学和语用学的边界在哪里?
\subsection{卡普兰理论}
Kaplan对于语境敏感性的问题提出了一个理论。为了引介这一理论对这些问题的解决方案,我们在这里给出一些(Kaplan的)名词的解释。

{\heiti 语境.}\footnote{它的决定因素一般地被认为是如下的四个方面:时间、地点、说话者(可能包括其目的)以及可能世界。}
一个“可能的使用的场景”。

{\heiti 评价语境.}
评价一个语句的真伪性的那个语境,换言之,可能世界。

{\heiti 内涵.}
一种通过给定的可能世界推导到真值的方式\footnote{这里的“内涵”和“意义”很相似。所谓的可能世界,我们完全可以认为其是上述“评价语境”;它从一个评价语境推向一个合适的延拓。举例来讲,作为句子而言,其内涵就是这个句子语义学地表示出来的内容。}(或者说,函数)。

{\heiti 指示代词\footnote{经典的Demonstratives是“that”“this”等;例如我们指向地图上的某处,我们就得到一个“here”。}(Demonstratives).}
一种索引性的表达,它需要一个相关联的指称,并且指代那个指称所指向的对象。

{\heiti 纯语义代词\footnote{举例而言,I,now,tomorrow等词汇。它不需要先天地存在某个指称才有意义,在任何语境下当我们说出这些词时都有意义,而不需要像使用“here”时一般指向某个确定的点。}.}
同指示代词不同,不需要任何的相关联的指称,但是执行索引性的任务。

内涵逻辑学家戴维·卡普兰(Kaplan 1977)把这样的规则看作函数。像内涵是从世界到外延的函数一样,一个语义——语用规则是从语境到内涵的函数。在句子层面上,内涵是从世界到真值的函数。卡普兰把内涵叫做句子的"内容",并日像以前一样,对应于传统的命题概念。合成的语义—语用规则是从语境到内容的函数,卡普兰把它叫做"特征"。

其基本想法是{\heiti 双参数方案}:我们首先确定语句产生的世界,之后再去考虑在每个确定的世界下这个名字的直接指称;我们将依赖这一点来考虑某个句子的语义。

\textbf{\kaishu 参考内容[6].}
{\kaishu {\heiti 两种考察可能世界的方式.} 查尔莫斯认为,存在“认知可能世界”和“形而上可能世界”。所谓的“形而上可能世界”指的就是一种可能却不是真的世界,即我们所说的反事实的可能世界。而认知可能世界指的是把可能世界当做事实世界来考虑。 例如,在孪生地球上,水实际上是XYZ。如果按照克里普克的理论,水是一个严格指称词,因而孪生地球上的XYZ不是水。这是形而上可能世界(在假定认知世界的事实性的情形下)相较于直接指称理论的一个问题。但是如果孪生地球是事实世界,则水仍然按照传统语义学来指称,也就是说,水就是XYZ。这就涉及到把什么世界当成中心的、事实世界的问题。因而,如果不预设一个事实世界,传统语义学仍然是成立的。这种从两个角度考察可能世界会导致两种不同语义学的方式,就是二维语义学。它在一个维度上,同Kripke的本质主义和直接指称没什么区别;而在另一方面上,我们决定它具体在某个可能世界获得指称时则有时需要一些描述性的特点,所以它事实上也是某种介于指称和描述之间的看法。

在可能性世界中,内涵是可以从外延推导出来的。这是因为,假设一个内涵X,在可能性世界W中,确定了一个外延Y。此时,可能性世界就是自变量,Y就是因变量,函项就是X。此时,如果两个专名的外延在所有可能性世界中相同,就能倒推出来其内涵也是相同的。

{\heiti 认知内涵和虚拟内涵.} 通过二维语义学确定的内涵同样也有两种,一种是认知内涵,一种是虚拟内涵。其中认知内涵是一种通过认知可能世界确定外延的函项,这种内涵依赖于先天的描述,接近弗雷格的意义。即一个人对某个表达式的认知理解。例如水的内涵,就类似于“清澈、透明、流动的东西”。虽然这个内涵不能被精准的确定下来,但是只要认知内涵只要足够在认知可能世界中确定某个外延就可以了。于是,在可能世界向外延的映射中,认知内涵作为一个函项,随着可能世界的不同计算出不同的外延。 虚拟内涵则是通过形而上可能世界确定下来的内涵。同认知内涵不同,虚拟内涵在每个形而上可能世界都指称同样的东西。例如水指称“$\mathrm{H_2O}$”,那么每个形而上可能世界所具有的虚拟内涵都能推出水是“$\mathrm{H_2O}$”。

{\heiti 指示代词与纯语义代词表述中的指称.}
就Kaplan的观点而言,“指示代词”对应的是demonstrative这一术语(特别是其中的简单指示词,即那些不带有后续名词的、非短语的表达式),其典型词项是诸如“这/this”、“那/that”这类词汇,对于由这类词汇构成的语句,例如,“这/this如何如何”,我们如果不在对应语境中识别出“这/this”所指称的对象(referent)是什么的话,我们就无法把握相关语句的语义内容,或者说,这类语词的元语义内容比较单薄,必须要在给定语境中找到其指称对象。这里,相关表述里所说的“ 依赖于那里所说的‘相关联的指称’”就是描述的上述现象,这些引文中所说的“指称”严格说来就是demonstrative在给定语境中的指称对象,即“referent”;而“纯语义代词”就是指以“我/I”、“你/you”等为典型代表的索引词(indexical),这类语词的元语义内容(或者用Kaplan的术语说,是其character)相对来说是较厚的,存在着相应的、可以独立于使用语境的元语义规则来判定相关索引词的指称对象。例如,在非引用的语境下,“我”就是指使用该语词的人类个体;这样一来,当我说出:“我如何如何”的时候,根据“我”的上述元语义规则来判定出“我”的指称对象就是
笔者,因此,“我如何如何”这一话语表达的命题就是“
笔者如何如何”,很明显地,对比“这”与“我”的语言功能实现方式,“这”与其指称对象(referent)的关联则明显突出,换言之,我们没有关于“这”的一般性元语义规则来判定某一语境中“这”的指称对象,恰恰相反的是,我们需要直接依赖指称对象来澄清“这“的语义内容。而“我”是依据相关元语义规则来判定其指称对象的,并不是“直接”通过指称对象来澄清“我”的语义内容的——“我”与“这”的语义功能的起效方式形成了鲜明的对比。至于引文“这些代词,无论是指示的亦或是纯语义的,都是指称的”中的“指称的”则是描述两类语词的语言功能的共性,即,referring或者to refer——注意:这里的“指称”是描述语词的语义功能的referring或者to refer的,这不同于前述引文中“指称”的那种严格说来是“指称对象”(referent)意义上的概念。

如果追求相对更严格的理解,对于诸如“这”、“那”等简单指示词而言,由于其元语义内容比较薄,我们也可以将其语义特证就看作是纯粹地或简单地指称功能(purely/simply referring),而我们这里说的“纯粹”或者“简单”就是说其“指称”是不能通过那种类似“我”的元语义规则来确立其指称功能的,这种情况下,指称/referring就只能是较为实质地依赖referent才能达成的了(此后的理解就可以参考上面的解说了),因此,对于“ 一个‘指示代词’的指称则依赖于那里所说的‘相关联的指称’”的“指称”就可以更严格地理解为上述我们所说的那种purely referring或者simply referring的指称功能。

{\heiti 二维语义学.}
在克里普克看来水作为$\mathrm{H_2O}$是必然的(虽然是后天必然的),水作为严格指示词在跨可能世界中都要严格的指示它的对象:$\mathrm{H_2O}$。在克里普克看来,一个既必然又偶然的命题显然是违反逻辑规则的,因而没有讨论的必要。但是查莫斯认为借助于可能性的形而上学与认知的区分,可以构造出能够成立的既必然又偶然的命题。在查莫斯看来,形而上学可能性和认知可能性这两个概念在外延上是不同的,水不是$\mathrm{H_2O}$是可以被无矛盾的设想的,因此是形而上学不可能并且认知可能的。查莫斯认为每个命题和表达式都是处于一定的认知者和某种认知情形之中的,因此每个命题和表达式的内涵是依赖于认知者的认知情形的,我们可以将这个情形理解为认知可能世界,它是认知者不能通过先天来排除掉的所有的可以无矛盾的设想的世界的所有可能状态的集合。由于认知者以自己所处的世界为现实世界,因此所有的表达式的内涵都要依赖于认知者所处的可能世界的情形,因此表达式的主要内涵也被称为认知内涵。而将所有的反事实世界与现实世界的集合看作是认知可能世界,次要内涵被称为虚拟内涵。通过对“情形”的论述我们可以解释一些可能世界语义学解释不了的问题。比如说一个水是XYZ的可能世界就是一个情形,在这个情形中,水是XYZ是必然为真的,但是对于另外一个(水是$\mathrm{H_2O}$的)可能世界,这个命题就是偶然的。因此“水是XYZ”或者“水是$\mathrm{H_2O}$”都是既(认知)必然又(形而上学)偶然的,只是把哪个世界作为实际世界不同罢了。

进一步地,解释这里的先天和后天。直接标示词既不是专名,又不是描述词。根据克里普克的说法,严格指示词(rigid designator)在所有的可能世界中都指称同一个对象,因此“晨星是暮星”就是后天必然的命题。但是直接标示词(index)所在的命题“我在这里”根据句子本身的特征,任何人说出这句话的时候,这句话都是必然为真的。虽然在任何的语境下都是先天为真的,但是根据句子指称的对象,在不同的语境中它指称了不同的对象,这就使相关的命题不是必然的命题了。因此,卡普兰区分了语言的表达的意义的两个维度-表达式的“特征”(character)和表达式的“内容”(content),前者是克里普克式的可能世界语义学的意义,后者是随着语境的变化不断地变化的意义。这样的依赖于语境的变化的意义就是表达的意义的二维的涵义。
我们有:

(1)句子S是先天的(认知上必然的),当且仅当S的主要内涵在所有情形为真;

(2)句子S是后天的,当且仅当S的主要内涵并非在所有的情形中都为真;

(3)句子S是形而上学必然的,当且仅当S的次要内涵在所有可能世界为真;

(4)句子S是形而上学偶然的,当且仅当S的次要内涵并非在所有可能世界为真。

认知二维语义学是一种融合理论。它通过对一些概念的重新的解读把些对立的观点融合在了一起。它一方面继承了描述主义的某些观点,一方面又继承了直接指称论和可能世界语义学的基本观点。试图在保留意义与模态的克里普克方面的同时,维护意义的弗雷格方面。
}

Kaplan的原则是:一个“纯语义代词”的指称依赖于语境,而一个“指示代词”的指称则依赖于那里所说的“相关联的指称”。这些代词,无论是指示的亦或是纯语义的,都是指称的\footnote{即是说,一个代词产生的那种内涵是它的指称,而不是什么别的感觉或者概念上的成分。}。

在Kaplan之后,产生了语境敏感性问题的哲学与语义学争论。Kaplan举出了一些依照上述“卡普兰风格”的指示代词。Kaplan指出,按照他的那种关于语境敏感词的论述,语义内容自然而然地要依赖语境去决定,存在着不仅仅蕴含于表面的语义表达的内涵和意义。这里的争论是:Kaplan的“指示代词基本集”是否已经涵盖了全部的语义敏感词?

解决这一问题的意见有三个学派。

{\heiti 最小主义语义学.}
这一学派认为,任何符合语法的那种自然语言的语句在不包含Kaplan风格的指示词的情形下它的语义都已经完整了。它强调,所有的语境敏感性都是由句法而不是别的什么驱动的;而由句法刻画的词语的敏感性只有Kaplan的那一些基本的敏感词。

{\heiti 索引词主义.}
这一学派认为接受一个语境敏感词必须包括某种句法驱动的成分,但是它认为上述自然语言当中事实上包含着隐藏的索引词\footnote{举例来讲,对于“It is raining”这一看似语义完整的句子,这一学派提出它的语言结构实际上是“It is raining in someplace at sometime”,这就涉及了两个指向性的索引词;我们必须给出这两个索引词的指称,才能够完整句子的语义。}。这就在保证语境敏感性来源于语法的同时,扩大了语境敏感词的范围;它将典型的语境敏感现象利用那种隐藏的索引词句法结构解释了。

{\heiti 语境主义.}
这个层面上的学派已经不再支持全体的语境敏感性都滥觞于句法;它认为所有语境敏感词还可以包含在若干语言使用的范式当中,亦即,语言表达式内。它指出,一般性的语句的字面意义都不能够表达一个完整的真值可评价的命题,必须由语境因素介入之后才能够使得一个句子真正地完整\footnote{例如经典的例子“John is ready”,最小主义语义学认为它已经足够,但是这一流派则认定只有在特定的语境下,John和ready才能被确定。}。

{\heiti 相对主义.}
这一学派反对在不同的语境下同一个语句将会具备不同的意义这一观点。它认为,一般的典型的不含索引词的语言表达,在语境改变的条件下都不会影响其意义(这被成为不含索引词语句的{\heiti 跨语境稳定性});而尽管如此,其的核心观点是设若有两个不同的说话者分别指出了一对相冲突的语句,这两个语句可能同时判定真值;其中的关键问题就是,语境影响的不是意义,而是真值条件的判定标准\footnote{也就是说,双方的语义不会因为语境产生任何不稳定因素,但是双方会对自己语句的信念(例如双方对某物的“大”的评价冲突,这里两个人对“大”的评价标准)依赖于语境产生不同。}(“Count As”标准),语境利用这样的方式在不改变语义的情况下依然能够影响语句。
\subsection{语义学与语用学的分界}
到这里,语义学与语用学的分界线是否清楚了呢?我们绍介三种不同的关于语义学同语用学的分界的理解。

第一种区分方法是,观察其是关于语言表达式的结构或内容的抑或是关乎说话者的语言使用的内容:这里前者表明其是语义学的范畴,而后者指出其是语用学的。这种理解的问题在于,它的这种分界是如此地模糊,以至于尽管难以被反驳,却没有什么实际的作用。它符合一般人们对这两种学说的差别的认识,却不能清晰地用于判断哪类问题属于哪个领域。

为了进一步清晰化这个判据,Paul Grice提出了更明确的区别方式,即标准二分法:研究语言在表述什么(What is said)的归属于语义学,而研究语言实际上在表明什么(What is implicated)的则归属于语用学\footnote{Paul Grice给出的例子是这样的:对于同一句话“I will be finished studying by 5 pm”,如果它是用来回应“What time will you be finished studying”的,则应该算作语义学的范畴;而用来回应“What time can I borrow your Biology book”时,则是语用学的范畴了,因为这涉及到说话者的语言使用,而不是某种仅仅通过研究两句话就能够清楚的关系。}。

近期出现的一个理论是R.Cartson指出的“语境敏感性”判据,对于语句保持恒定的偏于语义,而对于语境敏感的则偏于语用。但是这一分界并不被广泛承认。根本上,一个关于语言用法的约定使得一些表达式有固定的内容的问题并非是语境依赖的,但它同时也可以不归属于语义学。例如“And”暗示的时间顺序当然对语境约定固定,但是它仍然是语用学的范畴。
\section{意义是实体与意义是使用:对立的语言观}
意义本质上也被看作抽象事物,换言之,可称为"命题"。语句“雪是白的”是指雪是白的;同样,我们可以说它“表达”雪是白的这个命题。其他语句,甚至其他语言的语句,比如,“La neige est blanche”和“Der Schnee ist weiss”\footnote{这两个命题都是“雪是白的”的意思,但是表达不同。}表达同样的命题,因而它们都是同义的。这种命题论也适用于各种"意义事实",因为"命题"本质上是表示"意义"的另一个语词。但是批评家提出质疑,是否命题令人满意地说明了意义事实,或者它确实如此。在开始的时候,指称和意义的这些主题不是分开的,因为人们关于意义 所具有的通常的自然想法就是,意义就是指称。不过,我们已经批评了符合常识却站不住脚的关于意义的指称论,所以现在我们必须直接面对意义,并考察一些更精致的意义理论;同样因此,意义问题也可以被归结成为语言观的问题。本节的语言观对立体现在对意义的观点的不同,基本上可以被区分为实体论和使用论。
\subsection{意义抽象实体论}
第一种关于意义的实体论指出意义是某种抽象的实体。Frege的观点是,语句的意义就是思想,而思想则居住在“第三域\footnote{在Frege的晚年,他谈到,在外部世界和内在世界之外,还存在一个“第三域”,数、概念、含义、思想、真和假,以及其他抽象对象居住在其中。基本的想法非常简单。我们可以同意,至少存在两类事物——物理事物和精神(mental)事物。物理事物,例如桌子和椅子,树木和岩石,以及其他的经验对象,居住在外部世界中。精神事物,例如感觉,情感和“观念”,弗雷格也将其看作“意识的内容”,居住在内部世界中。但是,似乎还有第三类东西,包括你已经提到的那些东西——数、思想,诸如此类。这些东西不是物理对象;我们在外部世界中不能“感知”(就其字面含义而言)它们。但它们也不是纯粹的“私有的”事物;它们能够被不止一个人所理解。举毕达哥拉斯定理为例。我们全都拥有这个思想,即该定理是真的,所以这个思想不能是一个“观念”(按弗雷格对这个词的理解),它也不能被“感知”(按该词的本义)。所以,弗雷格论证说,必须承认第三域,以便容纳这些既非物质又非精神的实体。如Dummett所指出的: “弗雷格关于思
想及其构成涵义的看法是神话式的。这些恒久不变的实体居 
住在第三域’(The Third Realm),后者既不同于物理世界,也不同于任何经验主体的内心世界;
只要这样一种看法处于支配地位,一切都将是神秘莫测的。这是Frege反心理主义原则的构建部分。}”之中。而J.Katz则更进一步,提出“意义是自主的柏拉图世界中的实体\footnote{柏拉图认为:人们日常感觉到的个别事物,“总是变化不居的、不真实的” ,只有通过理性所认识的永恒不变的一般事物,才是真实的“绝对存在” 。例如,美的个体事物是可变和不固定的,它因人因时而异,甲说美、乙说丑,此时美、彼时丑,因而不真实,只有一般的美,才是真实的存在,才是实体。他说:“如果有人告诉我,一个东西之所以是美的,乃是因为它有美丽的色彩或形式等等,我将置之不理。因为这些只足以使我感觉错乱” 。在柏拉图看来,理念或形式是共同名字表述和界定的、若干或许多个体事物共同分享或分有的、不可被人感到但可被人知道的一般实体事物。简单地说,柏拉图认为存在着一个由观念组成的实在世界,即所谓“理念世界”。它是独立于个别事物和人类的意识之外的实体,个别事物是完善的理念的不完善的“影子”和“摹本”。}”。他指出,语义学并不是研究语言同世界的关系的学说;它研究意义本身是什么,同义性、意义冗余和意义之间的关系等;而指称理论至多是从意义理论里派生出的概念,意义同众多理念一起作为某种意义上更本原的抽象实体而存在着。
\subsection{意义心灵实体论}
意义心灵实体论坚持意义是某种存乎心灵的实体或观念。最为经典的心灵实体论来源于Locke,他指出语言表达式的意义就是心灵的实体\footnote{这是一种把意义现实化(Reification)的方案,它为意义找到了一个现实中的位置。},或者,某种程度上正是头脑中的观念。语言从而成为一种符号化的交流工具,借助语言,内涵于人的内在的思想成为一种可表达的对象。那么,语词就成为思想的组分(Components)的符号,而思想的组分实际上就是{\heiti 观念}(Ideas)。这样,语词实际上就意味着观念。观念,作为思想的组分,则被认为是某种心灵的意象(Image),它被认为是不能被他人感知(Perceive)的;自然而然,那些语词若被简单地理解成声音的记号,那么语词同观念的关系实际上就是任意的;它之所以有意义,并不是语词本身有意义,而是它在作为“心的意象”的可感知标识的情况下才产生出了意义。这同时说明,语词的意义坐落在心灵当中。这也是经典的意义观点。

然而,这一历史悠久的学说却面临着诸多的困难。主要困难包含以下的两个方面:其一是无法解释人的交流\footnote{根据Locke的说法,只有声音是确切可感的,而心的意象则完全不可感,这二者却只是依赖于一种“任意的”语词到观念的关系,我们根本不能从声音推导出什么观念,那么人们怎么能够用声音交流观念呢?这与Locke对语言的根本功能是“符号化地表达思想”这一说法有矛盾之处。},其二则是无法解释语词作为Component是如何组合成为思想的。因为,Locke所说的“语词承载所要表达的对象的组分”这一说法是相当通常的理解,语词是思想的“砖块”这一比喻也是容易接受的。但,语词是如何承载表达思想的功能的;或者,语句为什么不是单独的若干语词的堆积,而是奇妙地组合成了一个表达思想的工具。Locke只是模糊地指出这是“心的能力”,却没有一个进一步的解释说明它是如何组合而成的。

面对上述问题,Noam Chomsky则将语言直接指定成为心的能力(Faculty)。他指出,心灵本身仍然应该被按照生物学的方式被解释,而语言并非像经验主义者那里的解释那样,是某种习惯、社会约定、或者像美国的结构主义者\footnote{美国结构主义具有两个特点:
一是注重口语和共时描写;美洲印第安语很多都没有文字和历史的材料,这使得这一学派一开始就不得不从口语着手进行共时的形式分析。这种做法,跟欧洲学者着重书面文献的语文学和专搞历时研究的历史比较语言学都有显著的不同。
二是注重形式分析,避开意义这一因素。美国结构主义在结构分析中只注意可供验证的语言形式,不考虑心理过程,也不谈社会和历史等因素。在形式和意义的关系上,他们认为形式的对立能决定意义的不同。}认为的那样归纳地在经验的过程当中形成的,而是一种心的“机制”,产生于演化。其硬件是生理学的事实,而软件,作为语言的规则而言,是一种{\heiti 普遍语法}。这样,语言学就成为了一门经验科学,而不是什么通常意义上的社会科学;它研究人的语言能力(Competence),而不是语言使用(Performance)。语言学习的本质,是在个体当中实现上述的那种普遍语法,而经验在这里的作用不过是帮助我们在若干符合普遍语法的语言当中选择了一种具体的自然语言\footnote{自然语言(Natural language)通常是指一种自然地随文化演化的语言。例如,汉语、英语、日语都是自然语言的例子。}。Chomsky认为,行为主义的理论之所以不能解释语言学习的事实,是因为它对语言学习的理解根本就是错的\footnote{行为主义的语言学习模式坚持语言在本质上是刺激—反应联结,认为语言是人对外界―系列环境刺激的反应,可以观察和测量,可以通过外界强化、训练、塑造或模仿逐渐形成,是对外界环境刺激的习惯性反应体系;人的语言反应不具有目的性,完全受外界刺激和强化的制约。但有时,如果成人要求儿童模仿的语法结构和儿童已有的语法水平距离较大时,儿童总是用自己已有的句法结构去改变语言样本的句法结构,或者顽固地坚持自己原有的句法结构。此时,即使成人对儿童的语言错误进行不断的纠正,其结果也是无效的,而按照Chomsky的观点,这其实是实现来源于演化的普遍语法的过程,我们并不在乎什么经验;另外,行为主义者虽然承认语言是人类特有的,承认人类生理基础的作用,但却竭力贬低这种作用,因而被认为是从根本上抹杀了人类和动物之间的本质差异。}。语言学,同其他自然科学一道,通过建立理论假说去研究所面对的对象,而不是像结构主义者说的那样是通过归纳法。心理学实验从而对语言学具有重要的意义;人的语言直觉在这里成为了语言学的核心证据。

\textbf{\kaishu 参考内容[7].}
{\kaishu {\heiti 普遍语法.}
Chomsky认为,语言是创造的,语法是生成的。儿童生下来就具有一种普遍语法。普遍语法实质上是一种大脑具有的与语言知识相关的特定状态, 一种使人类个体足以能学会任何一种人类语言的物理机制及相应的心理机制。人类个体就是借助普遍语法去分析和理解后天语言环境中的语言素材。普遍语法是对于人类个体获得个别语法的共性原则的描写,它本身不会生成任何具体语言。普遍语法寓于个别语法中。儿童言语获得过程就是由普遍语法向个别语法转化的过程, 并借助于语言获得装置(LAD) 得以实现。UG理论认为语言能力是人类基因决定的,是天生(Innate)的。不管你说什么语言,处在地球上的什么位置,只要你是人类,你的脑子里就有一套唯一的、普遍适用的语法,人类的所有语言都可以抽象化成这样的一套语法。我们自然而然地知道哪些句子能说,哪些句子不能说,即便你只听过一遍:因为你的基因决定了你对语言的合法性的判断。

Chomsky说得很清楚:“普遍语法不是一部语法,而是一系列条件,用来限制人类语法的可能范围。”普遍语法是构成语言习得者的初始状态的一组特性、条件和其他东西。具体地说,普遍语法是一切人类语言必须具有的原则、条件和规则系统,代表了人类语言的最基本东西。人能学会语言,是因为人脑生来就存有人类一切语言的共同特点。这些共同特点就是普遍语法。由于语言原则是集体无意识的内容,所以各种族、民族儿童都生来俱有,也就是说有共同的语法;由于后天语言环境不同,即所获外界刺激不同,所以才在普遍语法范围内获得各不相同的母语生成语法。生成语法的作用是确定句子的可能范围。普遍语法的作用是用更概括的原则来确定语法的范围。某种语言的生成语法规则排除不合格句子,普遍语法原则排除不合格的语法,普遍语法是语法的语法。实际上,普遍语法是乔姆斯基为揭开人类习得语言的奥秘所做的假设,是假想的人类语言都要遵循的一系列抽象原则和必备的条件,是生成各种具体语言的基础体系,它不是社会规约出来的规则,而是人脑里的心智规则。乔姆斯基的基本思想就是要制定出少量的原理,通过很少几项原理的相互作用建立起整个语法机制,为各类语法提供一个普遍性的参照。}

Fodor则更进一步,它指出在语言背后存在着一种概念性的“思想的语言”,它较之具体的自然语言更为深层。他指出,这个“思想的语言”同自然语言形成的整体是一个“模块化”的结构,当然,这同样支持语言存乎心灵的实体,只不过较之普遍语法作了一个更深入的提取。
\subsection{语言的使用论}
与观念论相反,Ryle支持语言是某种行为倾向(Disposition)抑或是某种使用的对象,而不是某种实体,也不存在于任何的地方\footnote{Ryle比喻说,把语言看成心灵实体(例如传达某种希望)的情况下那种实体存在于Chomsky的“硬件”当中,就像是机器里的鬼魂(Ghost in machine)。而Ryle说,No ghost in machine.}。当我们表达我们希望什么或者相信什么时,这种表达形式似乎默认了存在着那样的“信念”或“希望”那种心灵实体,但Ryle否认这一点。他指出,这种情况下我们实际上是在表现一种行为倾向,当我们倾向于去做某事的时候,就会产生像是“相信”“希望”的看似是心灵实体的语言,但那其实是某种条件下我们倾向采取某种行为模式的表现。这种通过行为倾向去解释语言的意义的论调,告诉我们意义并不是等待我们去发现的实体,而是存乎实际的行为和使用当中。

\subsection{意义的使用论}
Wittgenstein及其追随者在后期支持一种含糊的“基本图像”学说\footnote{Wittgenstein在《哲学研究》第一节中通过来自奥古斯丁的引文引入了这幅关于语言的图景: 假定大人们命名了某个对象并且与此同时转向它,我看到了这 个事实并且领会到,这个对象经由他们想要指向它时所发出的那些 声音加以表示了。但是,我是从他们的身体活动——这个所有民族 的自然的语言——中获知这点的。(这种语言经由面部表情变化和 眼部的变化,经由肢体的动作和说话的音调来表明灵魂有所追求, 或有所执着,或有所拒绝,或有所躲避时所具有的诸感受。)以这 样的方式,我逐渐地学习理解了我一再地听到人们在其在不同的命 题中的诸特定的位置上说出的诸语词是表示哪些事物的。现在,当 我的嘴巴已经习惯于这些符号时,我便借助于它们来表达我的愿 望。Wittgenstein对这种观点的总结如下: “在这些话中我们得到了关于人类语言的本质的一幅特定的图像 ——我觉得事情是这样的。即这幅图像:这个语言的语词命名对象 ——命题是这些名称的结合。——在这幅关于语言的图像中,我们 发现了如下观念的根源:每一个语词都有一个意义。这个意义被配 置给这个词。它就是这个词所代表的那个对象。...我们也许可以说奥古斯丁的确描述了一个交流系统,只不过被我们称之为语言的并不全是那种交流系统。要是有人问:奥古斯丁的表述合不合用?我们在很多情况下不得不像上面那样说。这里的回答是:‘是的,你的表述合用,但是它只适用于这一狭窄限定的范围,而不适用于那个你原本声称要加以描述的整体’。”}。他们的看法是,意义并不作为任何实体存在,它被我们使用语言表达式时所存在的那种语言共同体的各种条件和规则完全决定。它取消语言的本质性,而承认语言的使用性;它认为意义存在于我们的使用当中,具备各种各样的使用方式,而没有清晰的本质在某处存在着使之能够被分析。

在这里,Wittgenstein否定了早期语言观里的一个概念\footnote{传统语言观认为,根据表达的意义来说,学会语言并成为一个有语言能力的说话者是这个说话者认识到语词同其指称的物体的关系的结果。Wittgenstein的观点是,语言的意义产生于使用。}:表达的意义在于它命名或指称的东西,句子的意义在于它的使真条件,换言之,某些通过存在保障句子的真性的可能的事实。在使用论那里,语言是一个社会的实践,奠基于人的社会活动;所以并不是先有某种关于语言的确定的本质(例如在观念论那里,语言的本质被认为是表达思想的工具)或者先有固定的功能(例如在描述论那里,语言某种程度上是用来描述事实的工具),而是从使用当中产生自己的意义的,它被用于描述、命令、许诺和恐吓等。词汇与世界的关系也是多种多样的,例如人的名字和那个对象之间有一种关系,而数到“五”\footnote{这是Wittgenstein的一个经典例子。他说:“现在,请考虑对于语言的这种运用:我派某人去买东西。我给 他一张纸条,在其上写有这些符号:“五个红色的苹果。”他带着这 张纸条来到杂货商那里;后者打开写有符号“苹果”的抽屉;然后, 他在一张表上寻找“红色”这个词并且找到一个与其相对的颜色样 品;现在,他说出基数词的序列——我假定他记住了它们——直 到“五”并且在说出每一个数字时他都从抽屉里取出一个具有那个样 品的颜色的苹果。——人们就是以这样的方式以及类似的方式用语 词进行运算的。——“但是,他如何知道应当在哪里和如何查找‘红 色’这个词并且他须使用‘五’这个词做些什么?”——好的,我假 定,他像我所描述的那样行动。解释终止于某处。——但 是,“五”这个词的意义是什么?——在此根本涉及不到这样一种意 义;在此涉及的仅仅是“五”这个词是如何被使用的。”在解释什么是理解“五个红苹果”这条表达式时,Wittgenstein在此讲述了人 们会对它做什么。对数字“五”的掌握并不是通过找出它所命名的某个独 一无二的对象而得以解释;相反,掌握是一件在某些支配其运用的特定 例程中进行的事。杂货商诉说了一系列声响 ——“一”“二”“三”“四”“五”——然后将它们与一系列动作一一对应起来 ——每个动作均涉及将一个苹果从抽屉中取出这件事。对这些数字的掌 握就是对这样的例程的掌握。这便是Wittgenstein关于意义就是用法这个主题的第一个例子,是其“语言游戏”的一个实例。}和它指代的那个抽象对象(如果有的话)也有一种关系。名字当然可以用于指称,但那既不是名字的本质或意义,也不是全体名字都同对象有着同样的指称关系。传统上,像是“This”“That”那样的最经典的指示词一开始并不是作为名字使用的,更遑论指称本身还有待澄清和分析。在不同场合下,指称是用来说明不同的事情的,我们不能用这样一个模糊而广泛的概念去作为说明一切语言现象的基础,更不必说拿它作为意义和理解的哲学分析的起点了。

同乔姆斯基的理论不同的是,Wittgenstein坚持我们在使用我们所理解的词语时,我们并不是依照某种普遍规则一成不变地被我们内省着掌握的规则所指导(这里,这些规则是用来决定我们使用词汇的正确性的);语言的使用,作为意义的来源,具有丰富的多样性, 从这一观点看来,语言并没有什么不变的对全体不同的使用都一致的本质;甚而,像我们开始时说的那样,可以说根本没有什么语言的本质。在讨论这个核心观念的过程中,Wittgenstein提出了两个相互联系的重要概念,叫做“语言游戏”和“家族相似”。

\textbf{\kaishu 参考内容[8].}
{\kaishu {\heiti “语言游戏”和“家族相似”.}
Wittgenstein在他的书中这样写道:“这是真的。——不去给出为我们称为语言的所有东西所共同具 有的某种东西,我说,根本不存在这样一种东西,它为所有这些现 象所共同具有,并且因为它我们运用同一个词来称谓所有这些现 象,——相反,它们彼此以多种不同的方式具有亲缘关系。因为这 种亲缘关系,或者这些亲缘关系,我们将它们都称为“语言”。我要 努力解释这点。”

他的解释方式正是上述“语言游戏”和“家族相似”。他是这样说的:“请考察一下比如我们称为“游戏”的诸过程。我指的是棋类游 戏,纸牌游戏,球类游戏,战争游戏,等等。什么是所有这些游戏 所共同具有的东西?——请不要说:“某种东西必定为它们所共同 具有,否则它们就不叫做‘游戏’了”——而是查看一下是否有某种东 西为它们都共同具有。——因为,当你查看它们时,你尽管看不到 某种为它们都共同具有的东西,但是你将看到诸多相似性,诸多亲 缘关系,而且是一大串相似性,亲缘关系。 我不能以比通过使用“家族相似性”这个词的方式更好的方式来 刻画这些相似性;因为存在于一个家族的诸成员之间的那些不同的 相似性就是以这样的方式交叠和交叉在一起的:身材,面部特征, 眼睛的颜色,步态,气质,等等,等等。——而且我将说:诸“游 戏”构成了一个家族。 同样,比如数的种类同样构成了一个家族。为什么我们称某种 东西为“数”?好的,这是因为它与人们迄今为止称为数的一些东西 具有一种——直接的——亲缘关系;而且经由此,人们可以说,它 便获得了一种与我们也如此称谓的其他东西的间接的亲缘关系。我 们扩展我们的数概念的方式有如在纺制一根线时将纤维一根一根地 往上拧一样。这根线的强度不在于任意一根纤维贯穿于其整个的长 度,而是在于许多根纤维彼此交叠在一起。 但是,如果一个人想说:“因此,某种东西为所有这些构成物 所共同具有,——即所有这些共同之处的析取式”——那么我会回 答说:在此你只是在玩弄字眼儿。同样,人们也可以说:某一种东 西贯穿于整个这根线,——即这些根纤维的无缝的交叠。”

这段关于游戏和家族的谈论在阐明如下论点的语境中被提出,即语 言或语言使用并没有本质。对语言的各种使用来说,没有一个它们都共 有的独一无二的事物,没有一个它们都必定符合的样式,而只有交叠在一起的诸多相似性的系统——就像在关于游戏的情况中那样。为了理解语言的一项独特的用法,人们必须按照它自身来对其进行检验,而不能 事先假定它必定符合某种单一的、先入为主的意义观。同其他大多数概念一样,一个 游戏的概念是含糊的。但这并没有使该概念有缺陷或需要某种依赖哲学 分析的精确化。同样,一门语言的概念也是含糊的,但这并不意味着该 概念有什么错误;也不意味着理解日常语言的方式就在于描述它与某种 完全精确的、逻辑上理想语言之间的关系。

总结来说,家族相似性理论(Family Resemblance)认为范畴的成员不必具有该范畴的所有属性,而是AB、BC、CD、DE式的家族相似关系,即一个成员与其他成员至少有一个或多个共同属性。范畴成员的特性不完全一样,他们是靠家族相似性来归属于同一范畴。而范畴没有固定的明确的边界,是随着社会的发展和人类认知能力的提高而不断形成和变化发展的。}
\subsection{语言的描述而非解释:方法论变革}
既然Wittgenstein已经推翻了语言的本质理论,那么我们就不必去寻找一种办法去解释语言这个客体;然而,我们在使用论的基础上应该用何种方法论对待语言呢?使用主义者们的观点是,在这里进行一个方法论的变革。我们假定在上述的概念下已经可以知道我们对特定语言使用的理解已经是本质的,必须就其本身而考察(而不是事先设想其必须适合于某个单一的符合所有其他实例的性质),那么我们就只是通过观察那些日常的语言使用,而完全不必企图通过语言使用的表面现象去解释本质——因为那个“本质”被认为是根本不存在的。

新方法的特点有如下三个:

{\heiti 诉诸日常语言.}
日常语言是语言使用的基本研究对象,我们不必抽象出一个虚空的语言实体,只需要诉诸日常对语言的使用来研究语言就足够了。

{\heiti 诉诸多元模式.}
根本上,因为语言是丰富多彩的,而我们已经再三地说明过使用论否定一个单一的标准的语言模式;所以要对语言有宏观把握,不宜再去寻求其本质的整体解释,而是走向对全体的具体的实例以及其发生的语境的分析。

{\heiti 诉诸人的实践.}
人的实践是对语言游戏的描述和运用。那么,“分析”这一概念已经产生了实质的变化;我们再也不必把它作为逻辑分析的代名词去寻找什么语言的逻辑形式了,不再需要寻求什么自然语言的谬误和语言表面形式下隐藏着的若干“真实”(像我们在摹状词和索引词那里做的那样)。我们只是描述,描述可能存在的不同形式的混淆,并揭示出那些混淆可能为哲学带来的混乱。
\subsection{言语行为理论}
这一理论是由A.L.Austin和J.Searle提出的。他们的观点是,只关心语言的用途(Penformance)。使用语言可以被认为是一种约定性的社会行动,它是一种“实用”的行动。每一个语句承载并实行一个语言行为,每一个语言的说出都具备着一种“语言外的力量”(Illocutionary force),而我们就关心这些语言能够被用来做什么(How to \textbf{do} things with words)。进一步地,这一学说认为言语行为作为一种受约定的社会行为由社会的规则、习惯或实践所支配。规则分为建构性的(Constitutive)和规制性(Regulative)的,前者建立一种规则(或者说,规则本身就定义一种体系),而后者则约束若干即成事实,对这部分规则的修改将不影响这些既成的内容。在语言行为那里,前者是一个行为成为语言行为的必要条件,后者则只是对行为的恰当性的限制。

Austin认为,语言是人的一种特异的行为方式,人们在实际交往过程中离不开说话和写字这类言语行为\footnote{言语行为有三种类型:语谓行为,即用词来表达某种思想;语旨行为,即说出的语句带有某种力量;语效行为,即利用说出一个语句来产生一定效果。要完成一个语旨行为必须通过完成一个语谓行为,因此语旨行为和语谓行为既交织在一起,又存在着界限,因为许多语谓行为并不同时起着语旨行为的作用。语旨行为和语效行为亦有明显区别,前者产生的效果是劝说性的,后者产生的效果是强制性的。}。语言分析哲学的中心课题应该是研究这种言语行为的本质和内部逻辑构造,在这里,言语行为,而非言语使用或言语本质,是意义和人类交流的最小单位。
\section{奎因和戴维森的意义理论:意义的规范性质}
Quine和Davidson对意义的认识,总地来说是站在一种整体主义的立场的。事实上,Davidson正是Quine的门生,从老师那里继承了整体主义和外延主义的思想。站在这一角度研究意义理论,我们将站在一个整体论的视角审视这整个意义理论。他们反对意义的实体论,主张“意义实体”的不存在性。然而,二者的意义理论也有区别:如果不能通过“意义实体”来直接判断翻译和理解的正确性,那么我们要依赖何种判据来确定意义?早期行为主义认定行为倾向就是那种语句的意义,而Quine(尽管身为行为主义者)去批驳了简单地使用行为倾向确定语句意义的认识,同时有自然主义\footnote{其将哲学划归为某种经验现象,指定哲学是“心理学的分支”;而人类身为自然存在物同外在的(经验的)世界交流的过程是哲学的根本问题。}的倾向。我们看待语言的过程被Quine认为是人类身为某种自然存在物最初始的把握基本的语言理解的问题,他通过指出基本理解的困难而说明了意义作为客体的不确定性。而Davidson并不继承其自然主义和行为主义的视角,提出了成真条件的意义理论。二者同时支持语言的外在论,但Quine承认部分的证实主义,Davidson(站在融贯论的视角下)则反对存在类似观察句的超越信念的证据。
\subsection{原始翻译及行为主义}
W.V.Quine进行了一个叫做“原始翻译”(Radical Translation)的思想实验。它的目的意在将可观察基础上针对语言翻译或者理解的证据基础揭示出来,考察在这个固定的条件下我们能将翻译或者理解做到何种程度\footnote{Quine在早期的工作是对逻辑实证主义者进行了批判。实证主义者虽然并不支持对意义本身的形而上学解释,但是支持意义有证实主义的定义。因此,我们可以引出经典的正确翻译的概念,尽管他们完全没有讨论这个问题。这是因为,逻辑实证主义者们主要讨论语言的逻辑问题,而不是行为和心理问题。}。

\textbf{\kaishu 参考内容[9].}
{\kaishu {\heiti 原始翻译思想实验.}当我们想到语言时,大部分时候我们想的是我们所熟知的语言中的词语。奎因认为,正是因为我们太熟悉自己的语言了,以至于我们看不清楚语言本身的本质。所以,他认为,只有研究一种我们完全不熟知的语言,我们才有可能了解到语言的本质。因此,他大胆地提出了这样一个例证:一个语言学家去一个他完全陌生的土著部落,既不了解那里的风土人情、民族风情,也不了解那里的语言。总之,语言学家对那个土著部落一无所知。那么,他将怎样来了解这个未知的语言呢?奎因的设想是这样的:当土著人看到一只兔子跑过去的时候,土著人指着兔子发出Gavagai的音,那么语言学家就把这个音记下来,姑且认为Gavagai就是兔子的意思。当然,要确定Gavagai到底是不是真正意义上的兔子,还是需要验证的。因此,当语言学家看到一只兔子从眼前跑过时,他指着兔子说Gavagai,如果土著人表示赞同的话,那么,Gavagai就是兔子的意思;反之,则不是。那么,语言学家的任务就是把目标语和源语言一一配对,制作成一部翻译手册(Translation Manual)。这种依靠观察获得的句子称之为观察句;而更一般地,存在着场合句\footnote{其意义与语境有关并带有索引词的句子。例如,“下雪了”或“今天是星期二”。它们在某些被说出的场合是真的,而在另一些场合则是假的。“这些句子的真值是随场合而变化的,因而每次都要求一种新的裁决。”是否能够接受这种句子取决于背景。当的确下雪了或是星期二时,我们才能同意这些句子。}和恒定句\footnote{这与语境无关,是经常并永远会得到赞同的。恒定句是对常识说法的分析或陈述,例如“雪是白的”。}。
由此可以看出,奎因是根据说话者对所表述的句子的赞同与否(即倾向态度)来判定语言的。奎因认为,不同的句子会引起不能的倾向态度,共有三种情况。第一种情况是,倾向态度随场合的不同而定,而不是随说话者的不同而定。比如,当我们指着一条狗说“That's a dog”时,我们都表示赞同;但是,当我们指着一条鱼说“That's a dog”时,我们则不会赞同说话者所说的。第二种情况是,倾向态度是随说话者的不同而定,而不是随场合的不同而定。例如,当说“God exists”时,有些人对此表示赞同,有些人对此表示反对。因为有些人是有神论者,相信上帝的村子;而有些人是无神论者,完全不相信上帝的存在。第三种情况是,人们对某些句子存在着相同的态度,要么是恒赞同,要么是恒反对。例如,在数学上,当说“2 + 2 = 4”时,我们都表示恒赞同;然而,当有人说“5比9大时”,我们表示恒反对。因此,语言学家的任务就是观察人们对某个句子的赞同与否的态度,然后把这些句子同本国语言中表示同样的倾向态度的句子一一配对。所以,奎因认为,句子的意义的根本就是对句子的赞同与否的态度。

但是,按照奎因的这种思路来的话,也会引起很多的不确定性。就拿上文所提到的Gavagai为例。当土著人看到一只兔子跑过去,然后说Gavagai时,我们不能确定他指的到底是什么。也许这个Gavagai指的是兔子的颜色,也许指的是动物,也许指的是兔子跑的状态,也许指的是兔子身体的某个部位,也许指的是雄性或者雌性,等等,这一切都是不确定的。因此,语言学家也没办法验证,因为当语言学家指着跑过去的兔子说Gavagai时,土著人也不确定他具体指的是什么。所以,这个事情就陷入了恶性循环。因此,奎因就提出了翻译的不确定性(The Indeterminacy Of Translation)。翻译的不确定性是于1960年奎因在他的著作《词语和对象》(Word and Object)中提出来的。奎因认为,导致翻译不确定性的原因有两个:意义或内涵的不确定性和指称或外延的不确定性。就如上文所提到的例子,语言学家都不知道Gavagai到底指什么,所以也就无法进行下一步的配对和翻译了。亦即,在连指称都不可知的情况下,我们是无法准确地翻译的。这就告诉我们,根本不存在什么正确的翻译,传统的翻译-意义理论自然也出现问题。}

奎因指出了若干可以翻译和不可翻译的(不确定的)情形。一个观察句、真值函项、刺激\footnote{一个句子S(对一个说话者在一个时刻)的刺激意义是这样一 对类——如果被询问的话会促使该说话者赞同S的那些情况的类(S 的肯定性刺激意义),以及如果被询问的话会促使 该说话者不赞同S的那些情况的类(S的否定性刺激意 义)。(Quine并不认为S的肯定性和否定性刺激意义可以穷尽当说 话者被询问关于S的事情时的所有情况;它们并不包括该说话者保 留或悬置判断的情况。)}分析性语句或主体内的刺激同义性问题(在被提出的情形下)\footnote{Quine的解释是:“比如,请考虑“Gavagai”。有谁能知道,该词项所适用的那些对 象压根儿就不是兔子,而仅仅是兔子的某些阶段或它在某个短暂时 刻的身体部分(brief temporal segments)呢?无论上述哪种情况 下,导致了对“Gavagai”赞同的那些刺激情况,与导致了对“兔子”赞 同的那些刺激情况都是相同的。或者,也许“Gavagai”所适用的那些 对象是兔子的所有的、各种未经分离的部分;而此时刺激意义也不 会有任何不同。这是否可以通过一些微小 的、补充性的指向(pointing)和提问来解决呢?那么,请考虑我 们该怎么做。如果你指向一只兔子的话,那么你便已经指向了兔子 的阶段、兔子的主要组成部分、兔子的融合物和兔性被显示出来的 那个地方。如果你指向兔子的主要组成部分,那么你同样已经指向 了余下的四种东西;对其他东西来说也是如此。对刺激意义中任何 没有被区别出来的东西而言,它都不能通过指向什么而被区别出 来,除非这种指向伴有关于同一性和多样性的问题:“这个Gavagai 与那个是相同的吗?”“我们在这里有一个还是两个Gavagai?””需要指出,最后那里的问题就是“刺激同义性问题的提出”。}都是可以解决的。但是,对于一般的语句,可以有满足可观察行为的,但不同(乃至彼此冲突)的分析假设(或者说,翻译手册),换言之,根本没有什么意义事实决定这些翻译手册中的哪一册是正确的。翻译的不确定性有三个来源:存在意义而没有识别意义的一般认知条件(或意义在认知意义下不可辨识);根本没有意义或意义事实的那种语句,其中它的语言或意义就是对言语行为倾向的总和;没有直观的意义或意义事实,但是对不同的翻译手册存在着间接的行为解释。这样,翻译并不归谬任何行为主义,或者,在整体论视角下,可以看出根本不存在什么不确定性。

针对行为主义的视角而言,其语言学习理论应当是具备以下四项的:其一是针对观察句的学习,因为在那里,行为主义的那种“社会条件下的刺激-强化”解释是学习得以可能的根本条件,而也正是这里实际指称的学习起到至关重要的作用。其二是分离指称,将单复数,冠词,指示词等等的对象同实际指称分离;其三是学习抽象的语句;其四是回归到学习与日常语言理解在模式上的一致性。

\textbf{\kaishu 参考内容[10].}
{\kaishu {\heiti 奎因的翻译不确定性与整体论.}
我们在上面提到过,关于Gavagai可能存在多种相容的翻译手册。即使Gavagai和我们所说的兔子有关,仍可能不等于“兔子”。它可能意谓某一时间段里的兔子,兔子的不可分离的部分,兔性,等等。
这里遇到的不是学习过程中的某种暂时障碍,而是原则上不可能澄清的东西。Quine早先把这种情况称作指称的不可测性(inscrutability of reference),后来改称为指称之无法断定(indeterminablity of reference),我们一般译作指称的不确定性。在讲这个故事之前,Quine就说明,用翻译来讲解不确定性原则最近于现实而不是那样抽象。翻译的不确定性论题说的是:“可以有多种多样的方法设计把一种语言翻译为另一种语言的翻译手册,它们可以都和土著言谈倾向整体相容,但互相之间则不相容。”所以,翻译不确定性有时也被称作“不相容的翻译手册”。在两种语言开始相互翻译的时候本来就没有手册,所以,翻译手册不相容性的根子就在指称不确定性。按照Quine的看法,没有一个更基础的事实来决定应该把Gavagai译作兔子还是兔子各部分组织的整体。这两本手册虽不相容,但可能都是有用的,甚至都是正确的,不能互相证伪,因为这里涉及的不是事实问题:“事实之成其为事实,这一点限于语词倾向本身”。Quine后来特别说明,翻译的不确定性其实是一个比指称不确定性更强的提法,因为它表示不仅在单词的层次上,而且在整句的层次上,始终都存在着不可调和的分歧。而指称不确定性主要是在单词层次上说的,可以设想在单词层次上有些不可调和的分歧,但这些分歧在整句层次上互相抵消了。而句子层面上的分歧只能通过其他一些句子在翻译中的分歧加以补偿。Quine曾经用维特根斯坦的一句名言“理解一个句子意味着理解一种语言”来提示不确定原则背后的整体主义。这的确有助于我们理解为什么翻译不确定性是个更强的提法。不过在另一个意义上,指称不确定会比翻译不确定更加严重,因为这会使你我用汉语交流也变成一种“翻译”,结果不再是通常所谓翻译不很确定,而是母语的意义本身就不确定。而这恰恰是Quine自己的看法。这就说明,整体主义观念在这里的观念是,在承认本体(指称)的相对性的前提下,单纯的语句或语词因为其不确定性都根本不能够成为意义的最小单位;正如我们在前面说过的,在整体论视角下,可以看出根本不存在什么不确定性——这里其实已经把最小的单位理解成理论本身了,只有这样才能彻底消除不确定性。任何指称都是相对于一个协调的整体才能成立。翻译手册的不相容性也指向系统的相对性,而不是要否认翻译的可能性。}
\subsection{戴维森的解释理论}
戴维森的解释理论的核心内容是真值条件语义学,并延伸地提出了{\heiti 戴维森纲领}。其基本思路是,一个语句的意义来源于给定语句是否为真的条件。在戴维森那里,真值条件语义学的基本思路涵盖下面的四点。其一是语句的意义由语句为真的条件给定;其二是一个小于独立语句的语言的表达的意义完全在于其对这个语句本身的真值条件的贡献\footnote{这就说明,对于若干有限的语义基本项和语法而言,它们必须且可能去构建无穷无尽的合法语句,若干小的语言表达的组织可以限制出无穷无尽的新的语句及其真值条件。};其三是,不同于逻辑实证主义的观点,真值条件只要求给定某语句为真的条件,而不需要了解语句实际上真值的条件为何\footnote{这里所要表明的是,不需要知道(像逻辑实证主义那样)实际上句子到底实际上是否成真,而只需要知道什么时候成真。比如就桌子上有一杯水这个命题而言,逻辑实证主义要求观察桌子上是否有一杯水,而Davidson的成真语义学只要求知道存在桌子且其上有且仅有一杯水的时候为真即可。};其四是,正像名字以其拥有指称世界中某个对象的功能同世界发生联系一样,语句是通过它的成真条件来同世界发生联系的。

在成真条件语义学的视角下,语义学的任务和解释的意义也相应地发生了转变。在Davidson那里,语义学的任务是构造一个具体的自然语言的真理论\footnote{在这种情况下,如何建立这样一个真理论成为一个语言理论需要考虑的核心问题。具体而言,就Davidson的观点,这问题可被表述为“什么样的问题是已经有了回答的”。},而解释就是明确一个说话者到底意指什么,换言之,辨认出语句真值条件的正确陈述;这里需要的则是语言行为和语言使用当中的经验证据。Davidson具体的做法是,进行了“原始解释”思想实验。通过这一思想实验,其意在解决的问题有如下两个:其一是我们能够知道的范畴是什么(这将允许我们解释说话者在特定的场景下所说的语句的意义),其二则是我们是怎样做到这一点的(亦即,给定了一个使得解释成为可能的理论,以及何种对潜在解释者合用的证据是在合理的程度上对这个理论进行支持的。)

\textbf{\kaishu 参考内容[11].}
{\kaishu {\heiti 原始解释思想实验和善意原则.}
Davidson在他的书中这样说道:“由于对有意义表达式的数目似乎没有明确限制,一个可行的理论必须解释那些建立在表现出有限特征的模式化基础之上的表达式的意义。然而,即使对这些句子的长度有某种实际限制,一个人仍能够带着理解地发送和接收,需要一种令人满意的语义学来解释可重复特征对于它 们在其中出现的句子意义的贡献。”这里,他求助于我们理解长而陌生的语句的能力,并提出对这种能力的一种解释。在我们有限的词汇及有限的语言经验的基础上,我们如何理解潜在的无限多的英语句子呢?答案一定是,我们已经掌握了"有限数量的特征",一个相对较少且易于控制的那些作为意义"原子"的意义表达式的集合,我们也掌握了一些组合规则,即把这些原子或原始语义组合起来的"模式化"方法,而那些原子或原始语义会产生更复杂的表达式的意义。

非常粗略地说,意义原子是单个的语词,而组合规则是详细说明语词如何能结合起来从它们的个别意义映射到更复杂意义的语法或句法规则。戴维森主张,一个语句的意义就是构成它的那些语词的意义的一个函数。这正是组合性的论题。对于说明我们对长而陌生语句的理解,组合性是一个显而易见的假设:我们通过合乎句法地把语句分解成较小的有意义部分,以及根据关于这些语句的最小意义部分的句法函数,计算复杂的意义,来理解复杂的意义。所以,在通常的哲学意义上,一个充分的意义理论应该能为我们在建构相对于任一给定语言的系统的"意义理论"方面提供指导,通过由构成这个语句的语词来叙述其组成成分,这一理论将详细说明相对于那种语言的每一个合语法语句的意义。因此,就会以这种方式形成一个列表:
“雪是白的”意指雪是白的。“草是绿的”意指草是绿的。“笔者在2016年就任北大校长”意指笔者在2016年就任北大校长...

而这一列表是无穷的或者说是潜在无穷的。当然,这个例子在英语中详细说明了英语句子的意义(而这听起来有点令人厌烦),但对其他语言我们也必须能够做同样的事情。一个关于英语的理论或者一个关于德语的理论何以可能形成这样一个列表呢?首先要注意,如果我们知道足够多的事实,相应于我们理解长而陌生语句的能力,我们具有判定这些语句的真值的能力。长句子的成真条件由组成它们的较短句子的成真条件决定,而且形成长句子的句法过程随身携带了与真相关的语义性质,从而由简单的真性质组合成更复杂的真性质。在《当代语言哲学导论》那里,威廉·G·莱肯举出了一个这样的例子:其描述了一种非常简单的小语言Oafish,几乎像维特根斯坦的建筑工人的语言一样简单,不过这一语言具有根本不同的特征。它有两个词项式调词F和G,分别对应于语词胖和贪婪。F指称或适用于所有而且只有那些这个世界上胖的事物,而G适用于所有而且只有那些贪婪的事物。这个小语言也有两个专名:a,它指称阿尔伯特;b,它指称贝蒂。它有一个形成主谓句的语义规则:由在专名 n 前面加上谓词 P作为前缀的句子是真的,当且仅当,n的指称包含在P所适用的那些事物之中。最后,Oafish 包括两个称作语句联结词的进一步的表达式;并非,它可以插入到任一给定的语句中;并且,它可被插入到两个完整句子的中间以形成一个更长的句子。这些联结词中每一个都受不同语义规则的支配。并非的规则是,由在另一个语句A之上加并非得到的语句是真的,仅当 A本身不是真的。合取的规则是,像A并且 B这样的复合形式的语句是真的,仅当A是真的并且B也是真的。

这就是整个语言——它的全部词汇、它的所有的任一种意义规则。它只具有有限的意义,并且它鼓励乏味的重复。但它的真定义,甚至是在它最愚蠢的简单性上,都具有我们所需要的相似特征:它考虑到无穷长且无穷多的合乎语法的 Oafish 语句,而且(尽管如此)它设法说明其中每一个语句的成真条件。例如,如果一个 Oafish 说话者说出Fa,那么我们就由这个主谓句可知,该语句是真的仅当a的指称,即阿尔伯特,包含在F所指的事物的类中,即胖的事物的类,这句话恰好是说阿尔伯特是胖的。(一个词项所适用的事物的类被称为该词项的外延。) 或者人们可以说,阿尔伯特是贪婪的。或者人们可以说,他是胖的并且他是贪婪的,因为关于"合取"的真值规则告诉我们,“Fa 并且 Ga”是真的仅当阿尔伯特是胖的并且他是贪婪的。而"并且"这个词可以重复,它可以被不断重复地使用,从而形成长而又长的没有停止的语句:Fa并且并非 Fb、Fa 并且并非 Ga并且 Fb 并且并非 Gb、Fa并且 Ga并且并非 Fb 并且 Gb 并且 Fa并且并非 Fb,等等,无穷无尽。(当然,后面的语句中充满重复,因为 Oafish 拥有如此少的词汇,但即使是有最多重复的语句仍然是合语法的,而且具有完全清楚的成真条件。)因此,仅仅从这个微不足道的小的真定义,我们就已得到无穷多合语法的语句,而且我们有这样的映射规则,它告诉我们一个句子为真的条件,不管该句子有多长。带着这样的装备,我们就能够处理任一陌生的 Oafish 语句,即使它有五英里长,并能够计算出其成真条件。我们已经根据有限的、实际上很小的方式,解释了一种潜在的无穷的能力。这就是戴维森解决理解问题的方式。他摒弃了行为主义的观点,而是指出了一种高度形式化的解释方式。 

就互相理解的一点,站在融贯论\footnote{知识论里面关于信念辩护结构的一种理论。它认为信念构成了一个网状的结构,互相提供辩护。}的立场上,Davidson指出他的理论不仅需要真原则和客观性原则,还需要某种善意原则;在成真条件语义学那里,释义理论是真理论。我们通过如下方法,在经验上证实一种被 提出的真理论——该理论是对一组给定的说话者的语言而言的—— 是正确的:比较那些说话者认为自己语言中特殊的句子在其中为真 的那些条件或情况,和被我们所测试的理论分配给那些句子的真值 条件。在所有其他因素相同的情况下,对一个给定共同体的语言来 说正确的真理论是这样的,即根据该理论,说话者实际上认为句子 在其中为真的那些条件,最接近于符合如下条件:在这些条件下, 该理论与我们关于世界的理论的联合,预言这些句子为真。概言 之,正确的理论是这样的:与依据其他任何对该语言的释义相比, 在依据这种理论时,讲这种语言的说话者都更频繁地成为讲出真理的人。戴维森认为从上述标准可以得出:我们释义另一组人的言语的能 力,涉及将他们视作是与我们共享所有被我们当作为真的信念。他大概 认为,这涉及如下事情:将他们视作是与我们共享了更多的真信念,而 不是采取任何其他可以归属给他们的可供选择的释义(假定所有被考虑 的释义都恰如其分地是综合的,并且遵从简单性的惯常原则和好的科学方法论)。这与他的论证的相关之处如下:如果 其他概念框架(看待世界的基本方式)可能与我们自己的完全不同,那 么——假设它们能够在某些语言中被表达——要么那些语言可以被翻译 成我们的语言,要么不可以。如果它们可以翻译成我们的语言,那么在 我们和讲那些语言的说话者之间一定有大量的共识存在,在此情形下, 他们不可能拥有与我们迥异(radically at variance)的概念框架。所以, 如果有与我们迥异的概念框架存在,那么我们将无法释义或翻译那些分 享它们的人的语言。假定L是这样一种语言。和所有丰富到足够表达系 统的世界观的语言一样,L会包含大量的真句子——其中的大部分将可 以被翻译成我们的句子,仅当作为一个整体的L(大部分)是可翻译 的。这意味着,如果L并不是可翻译的,那么它必定包含很多我们不能 翻译的真理。但戴维森坚称,这种想法是不融贯的。他论证道,我们对 被运用于另一种语言中句子的真这个观念的把握,与我们将那种语言翻 译成我们语言的能力紧密相连。因此,他认为,一种对真这个概念的恰 当理解,揭示了如下假设是不融贯的:在任何一种(大部分)不可以翻 译成我们语言的语言中,有很多真句子存在。他因此得出结论说,有理 性的施事者们可以拥有与我们迥异的基本信念——可替代的概念框架 ——的想法是不融贯的。这就是善意原则的观点。我们指出,戴维森在真概念上是融贯论者,因为Davidson认为真概念是相互提供辩护的底层逻辑,它们本身已经不再存在什么更深层的东西去论证了。
}

总结地来看,于Davidson而言,上述其对于语义学的理解(事实上,Davidson对语义学的观点就是所谓的Davidson纲领)可以被总结如下:一种自然语言的语义学本质上采取Tarski式的真理论,而那种真理论将蕴含所有形如“S is true if and only if P”的模式(这被称之为T-约定),后者是语义学理论的元理论,指定了好的语义学的形式。站在真理论语义学的起点上,为了构造那种真理论的证据成为Davidson意义理论的核心问题。其在这里的观点是,原始翻译的观点并不局限于对未知语言的解释,它甚至深入到一般性质的语言(包括自然语言乃至母语)的交流当中。首先,(带有先验论色彩地)应当接受人与人之间交流的可能性,站在这一实际的基础上去解释交流之所以成为事实的条件\footnote{这就成为怀疑论的一个反驳。怀疑论的观点是,反对某一个关于交流成立性质的判据;而事实上先验论指定了语言交流的成立,假定推翻某个判据就会顺势推翻实际上已经成立的语言交流。其认为,信念而非因果才能成为语言的证据。}。Davidson提出的结构是语言上的一个三角形,用以取代翻译和理解的意义判据\footnote{事实上,正确的翻译在传统视角下承认意义实体的存在;我们在翻译的过程当中假如正确地得到了这个意义,就意味着翻译或者理解的成功。
而Davidson的观点是,不仅仅是原始翻译的情形,任意的交流都基于下面的三角形,是一种统一的模式;原始翻译是通过削弱对母语的熟悉性来阐释这一统一性的描述,依赖于那种因果三角才使得互相理解成为可能。}。其中任何一环都是不可缺少的,否则将不能支持交流的成立。
$$\xymatrix{
\mathrm{Communicator\ A_1} \ar[r]\ar[dr] & \mathrm{Communicator\ A_2}\ar[l]\ar[d]  \\
 & \mathrm{Certain\ Object} &
}$$$$Figure:\text{Davidson的三角形交流结构.}$$
Davidson这样阐述(更基本的)关于原始解释和理解的条件:

{\heiti 善意原则.}
尽可能把被解释的语句看作符合于我们的逻辑,换言之,按照我们的原则是真的。

{\heiti 真理论框架.}
真理论是信念和意义的基本框架。

{\heiti 先验论证.}
一定存在着共同的而且最基本的对世界的理解,以及某种与世界因果作用的{\heiti 共同的方式},这要求非理性和基本信念上的普遍性错误是不可能的。

{\heiti 因果三角.}
正确性和客观性的基础,是人与他人、人与世界公在的因果联系决定的\footnote{但是这里不承认因果三角是证据的基础,而只承认其是使得理解成为可能的基础。交流当中的信念证据只有一个,就是信念本身,而绝不是什么因果。}。

\textbf{\kaishu 参考内容[12].}
{\kaishu {\heiti T约定和Tarski真理论.}
T约定是指,S在语言L中为真,当且仅当p。其中,S是对象语言L中的语句,p是S在元语言中的翻译。元语言中的语句被理解为直接谈论物理世界的事实。

上述 “S is true if and only if P” 是Tarski的著名的"T约定"的一个实例,远非Tarski真理论的全貌。事实上,“真理论”这个词是容易引起误解的,因为说到真理论,一般会联想到符合论、融贯论、冗余论等等,Tarski的工作并不是这些,他实际上是在定义“真”这个概念,所以说“Tarski的真定义”比说“Tarski的真理论”更容易把握。要定义“真”,也就是要 给出一套规则,说一个句子真当且仅当它满足什么样的条件。 “真”是一个语义概念,Tarski的工作实际上是建立一套语义学。

首先,说一个句子真,这个“句子”是什么?由于这是在建立语义学,因此定义中当然不能包括什么语义成分,而去掉了语义的句子不过就是按照语法规则排列的符号而已。而符号总是相对于特定语言的,不同语言会有不同的符号,因此句子必须是某个语言中的句子。而自然语言那么多,不需要给每种语言都建立一个语义学,只需要给它们共通的部分——深层的逻辑语法——建立一套语义学,而这实际上就是给形式语言建立语义学了,也就是模型论的工作。Tarski工作的主要部分就是给一阶语言建立形式语义学。然后,“真”的定义其实就是说一个句子为真当且仅当一定的条件成立,这个条件既然不能出现语义概念,那该用什么去定义“真”呢?这就要说到Tarski的物理主义\footnote{粗略地来说,物理主义就是说,所有概念能够最终完全通过数学、逻辑和物理学的概念来定义。其中数学和逻辑是没有什么争议的,所有人都会同意它们的基础地位,物理主义的核心就在于认为物理学、物理概念、物理实体可以比其它概念具有更基础的地位。}立场,Tarski就是要将“真”这一概念建立在物理概念上,将其它所有语义概念(有效性等)都建立在“真”概念上,从而建立一套语义学(Tarski的科学语义学),将所有语义概念还原为非语义概念。

在参考内容\textbf{[11].}的末尾那里,我们提到Davidson指出了Tarski的真概念的底层性。这与Tarski的观点有相同之处:形式上来说,Tarski证明了,在他对“真”的定义下,如果一个形式语言丰富到包括了自然数算术系统,那么就可以使用和哥德尔不完全性定理一样的证明手段,证明这个语言中不可能包含一个谓词T刻画了“真”,如果包含这样一个谓词的话,那么由于包含了自然数算术系统的语言具有自指的能力(以哥德尔编码的方式),那么就可以仿照哥德尔用形式语言构造哥德尔句“这句话是不可证的”一样,用形式语言构造“这句话不是真的”。这就是著名的“真的不可定义定理”。元语言和对象语言的区分就是从Tarski这里开始的,他认为“真”应该是一个元语言谓词,用来描述对象语言中的句子的真假。元语言当然可以和对象语言是同一个语言(或者更精确地说,元语言可以在对象语言中翻译出来),但这种语言因为表达力过于丰富而具有了自指的能力,因此“真”谓词在这种语言中无法定义。只有在表达力较为贫乏的语言中,元语言比对象语言更丰富,才能摆脱自指,给出实质适当的——能推导出T约定的全部实例的——“真”谓词。

{\heiti Quine的怀疑论和Davidson的反驳.}
某种程度上,可以认为Quine和Davidson的一个相当重要的根本区别在于是否具备着那种极端的怀疑论。于Quine而言,他是语言意义的极端怀疑论者。在《两个教条》当中,其针对经验论和逻辑实用主义提出了这样的反驳,即意义事实根本不存在。蒯因首先否认我们的"意义事实",并根据他的"翻译不确定性"原理而极力主张意义取消论或意义虚无主义。我们当然希望在这里彻底地阐释Quine对分析性和同义替换的看法,但是把全部的引文放在本笔记内显然是不现实的,故我们仅仅简单阐述这里的问题:他主张(通过翻译不确定性等)走向意义的虚无主义,并且转向自然主义和实用主义(典型的问题像是观察句)。他对我们理解自身和我们语词的日常方式之核心信条的批量拒斥,并没有任何坚实的基础。他关于 我们日常意向性观念——意义、指称、信念等——的极端怀疑论,建基 于他对翻译不确定性的论证,以及随后对该论证的扩展。而Davidson的解释理论则拒斥那种怀疑论,为意义赋予了实际的内容。戴维森在《关于真理和知识的融贯论》一文中也对怀疑论给予了这样的描述:它是一种怀疑我们具有任何“关于现实世界”的知识的观点。它承认所有“我 (关于世界)的信念是统一的”。 它质疑是否有充分的理由认为融贯的信念是真的。如果戴维森的“宽容原则”要对怀疑论的这种主张进行反驳的话,那么它就必须能够证明我们关于现实世界的信念的确是真的。通过补充真理论框架和先验论证、因果三角的方式,Davidson指出了信念这个底层证据,并依据此指出怀疑论的问题在这个视角下根本无从提出:必须具备融贯的信念,才能够产生(不是普遍错误)的交流和理解。
}
\section{形而上学:历史与基本问题}
何为形而上学(Metaphysics)?事实上,形而上学包含着广泛的哲学领域和问题。一般地来讲,与任何事物的存在,本质和基础相关的都可以算作形而上学的范畴,它的边界看起来是较为模糊的。然而,哲学家们大体对某个问题是否隶属于形而上学有一个基本的共识;举例来讲,它包含事物的存在性、世界的基础结构和本质、存在物之间的基本关系、世界的构成性组分的描述等。

\subsection{分析的形而上学及其历史}
上个世纪五十年代以前存在着占据主要地位的反形而上学的逻辑实证主义;传统的形而上学问题则遭受分析哲学家的排斥,某种意义上是无意义的问题。然而,在Quine的《经验论的两个教条》之后,逻辑实证主义遭受了反驳,形而上学的标准被重新引入了分析哲学的研究范畴,而原始翻译那里带有实证主义色彩的观察句仅仅被作为融贯论的“信念之网”的一个边界。这时,一种结合经验主义的实用主义开始兴起\footnote{Quine的《两个教条》的开篇语这样阐释:“现代经验论大部分是受两个教条制约的。其一是相信在分析的或以意义为根据而不依赖于事实的真理与综合的或以事实为根据的真理之间有根本的区别。另一个教条是还原论:相信每一个有意义的陈述都等值于某种以指称直接经验的名词为基础的逻辑构造。我将要论证:这两个教条都是没有根据的。正像我们将要见到的,抛弃它们的一个后果是模糊了思辨形而上学与自然科学之间的假定分界线。另一个后果就是转向实用主义。”}。进一步地,Strawson的描述性的形而上学则试图探索我们的日常语言和思想背后的基本假设。

到了二十世纪的六七十年代,心灵哲学重新兴起,与形而上学关系紧密的“身-心问题\footnote{心身问题(mind-body problem)是心理学最根本的理论问题之一,是哲学中的心物问题在心理学中的体现。它讨论心理过程与躯体过程特别是大脑的神经生理过程间的关系。}”重新回到了哲学舞台;到Kripke在讲演之中开始重提本质主义、可能世界的性质和后天必然性三个方面,这都是经典的形而上学命题(尽管其导出的结果同经典形而上学是不同的)。彼时,形而上学似乎仍然是模糊的,具备一定的工具性,而没有成为很主流的分析课题;对于后面八十年代的哲学家们而言,那些对于形而上学的批评似乎已经模糊起来,使得他们可以从一个新的起点出发去对待形而上学,从本体论等的角度出发去继续建构形而上学。而到目前为止,几乎全部的经典形而上学主题都成为分析形而上学的合法主题;同时新的形而上学课题也被开发出来,例如可能世界理论等。

而到目前,这里有两类人否认将形而上学作为某种知识学科\footnote{典型的形而上学信念认为,如下的内容的研究是值得从事的:世界的普遍真理,或者某种世界的本质特征或基本结构等}:历史或文化的相对主义者\footnote{其观点是,不存在世界的一般本质或者结构,而是必须依赖于文化或历史的共同体去认识。}及在认识论上秉持自然主义\footnote{主张用自然原因或自然原理来解释一切现象的哲学思潮。其观点是,人是自然存在的经验主体,除却观察边界,一个理论不应该蕴含世界本身有什么本质,而是蕴含人与世界发生作用的方式,具有外在主义的倾向性。}的那些人。
\subsection{形而上学的典型问题}
第一个典型问题是关于因果性及其本质叙述的。这里的问题是,是否每一个果必然地指向某个因,或者说,因果律是否必须本质地被认为是某种准则?就这个问题而言,休谟和罗素的回答是否,而大多数哲学家承认这一点并依赖概念去回答具体为何。在知识论的领域里,哲学家试图利用这一点来解决Gettier的反例\footnote{一直以来,西方哲学界对知识的定义包含了三个要素,即所谓的得到辩护的真信念,英文中常被简称为JTB理论。具体来说,某个人A“知道”某个事件B,或说A掌握了关于B的知识,是指:B本身是真的;A相信B是真的;.A相信B为真是得到辩护的(或者说有理据、合理的或确证的)。这样的情况下,获得的知识是真实可靠的。
JTB理论中的每一点都是必要的。葛梯尔的反例是:
史密斯被告知琼斯有一辆福特车,他因此相信这件事,并同时也有理由相信:“或者琼斯有一辆福特车,或者布朗在巴塞罗那”,虽然他根本不知道布朗在哪里。事实上,琼斯并没有福特车,但是布朗的确在巴塞罗那,所以史密斯相信的事情是真的(真信念),并且是得到辩护了的,但并不是知识。外在主义认为,使得A有理由相信B是因为客观存在的外部因素。葛梯尔的反例似乎有一个特点:使得A有理由相信B的原因客观上都不是真正的使得B为真的原因。比如,使得史密斯认为“或者琼斯有一辆福特车,或者布朗在巴塞罗那”是真事的原因是他有理由相信“琼斯有一辆福特车”。而事实上真正使得这句论断成真的原因是“布朗在巴塞罗那”。如果这两个原因是一致的,就可以避免这种情况了。因此阿尔文·戈德尔曼(Alvin Goldman)提出:知识是通过恰当的因果关系产生的真信念。也就是说,真实的事实B,经过一系列切当的因果关系后,成为了使得A有理由相信B的原因。};在语言哲学的方面则用于解释语言是如何联系到其指称的(例如,Kripke的因果链条)。

研究因果作用首先将性质分划为范畴性质\footnote{范畴指种类的本质。它不是种类本身,而是用来对事物进行分类的性质依据。例如“圆的”。}和倾向性质\footnote{指能够发生的一种倾向性。例如“可燃烧的”等。}。对于后者,我们描述其的时候一般地使用条件式(一般地,虚拟式或反事实条件句,而非陈述的),关于那种条件式蕴含的因果律将导向对自然或因果法则的思考。这里,因果论同以下四个内容相关:“因果力量”(假如存在),反事实条件的成立性,自然法则及实际变化的产生范式。其中何者是更基本的以及其中的关系成为形而上学的重要问题。

第二个典型的问题是时间性,这里研究的是时间的本质和持续性(亦即,某一确定经历过时间的对象的存在性)。相关研究存在A理论和B理论的分野;A理论指出时间是流动的且具备动态性,而B理论认为时间不是流动的,它承认“现在”这个语词的指称性仅仅指代表述的那个时刻(从而过去现在和未来某种程度上成为同样真实的);关于时间在物体当中的持续,我们的观点分为持续主义和恒定主义,前者承认持续存在的物体在每个存在的具体时刻都完整地存在,后者则不承认某个实在的对象,只承认对象在每个特定的时刻上存在着一个特殊的不相干的临时切片,而这些切片才是基本的。

下一个问题是构成、结构和同一。这里的问题包括物体的性质是否在拆分和组合的意义下保持,或者已经改变等。具体地来讲,这些概念典型地引出如下问题:

{\kaishu 第一个问题.}
一个桌子同占据桌子空间里的那些构成的电子和原子是否是同一个物体,后者是否单独于桌子而存在,亦或者它们是否是同一的(尽管其在描述上似乎等价)?

{\kaishu 第二个问题.}
某一个雕塑和构成它的那些泥是否是同一个物体,或者雕塑和泥是只存在一个还是都单独地存在?从泥到泥塑的转变过程是何种时刻完成的,判据如何(尽管在物理上是完全一致的)?

{\kaishu 第三个问题(The Ship of Thesues).}
将船体分别拆解成若干片,此后运用这些旧有的船板去重新组建一艘船,这两条船是同一条船吗?若船体每当有一块板变旧就更换之,直到全体船板都更换为新船板时这两条船是否是同一条船?进一步地,假定对某船A,按照上述问法逐渐拆卸旧板替换新板,全部替换为新板的船和用替换下来的旧板制造的船哪一艘是船A,证据如何?

上述问题的结合和交织又产生新的问题。包括分裂与融合的问题\footnote{例如桌上有3个苹果;那么桌上的完全独立的对象是三个,还是$\mathrm{C_3^1+C_3^2+C_3^3=7}$个?换句话说,这些个别的组合是否独立于体系而存在。}、内在和外在变化的问题\footnote{例如某种外在的关系性质,如你的某个朋友打了喷嚏,就使得你忽然获得“打喷嚏者之友”的性质,这种变化被称为剑桥变化,性质成为剑桥性质。这不改变某人的本质性。据此,来考虑内在性质和外在性质的区别。}、人的同一性问题\footnote{例如是否存在一个统一的“我”,抑或是我由很多时间切片构成,包含“$\text{我}_{t_1},\text{我}_{t_2},...$”,而每一个都是单独无关的?}、自由意志的存在性\footnote{经典的讨论是“海战问题”:明天必然会发生或不会发生海战。如果明天将会发生海战,那么海战是不可避免的结果,反之亦然。这样,我们认为,全部的未来发生的事情都将成为不可避免的,自由意志将失去其存在的位置。}、殊相和共相(一般或个别)、具体对象和抽象对象问题\footnote{事实上,存在着一些抽象的对象,而一般(共相)的内容必然对不同的抽象事实所共享。那么,是否存在着这样的共相为一些殊相所共有,亦或者它们分别独立?假如共相是存在的,它们是什么样子的,我们如何界定一个从众多殊相中抽象出来的实体?又例如,具体的抽象对象(如赤道)在何种意义上存在?}、本体论承诺、事件与事态问题等\footnote{一个典型的议题是,某一事件的原因是某个主体产生的原因,亦或者别的什么事件或事态的结果;换言之,来自于某一主体的动因乃至于能动的主观精神动因是否存在的问题。另一个典型问题是我们是否应该把事件纳入本体论对世界和事物本质研究的范畴内部?一个行为又是否可以被纳入事件的范畴?}。
\section{模态形而上学:可能与现实}
模态的形而上学研究的问题是“可能性”、“必然性”和“偶然性”这种观念等问题。就关于的对象的问题,模态可以区分为从物的和从言的,前者关于的是某种对象,后者则更大程度上关于语言表达式、或者关于某种对象的描述、表达和陈述等的命题\footnote{举例来讲,从物模态的问题像是“Nixon是否必然是美国总统”,而从言模态更像是“<Nixon是美国总统>这一命题是否必然为真”;描述论视角下,后者语句的真值同这个对象被描述的方式有关系。T.Sider的例子是:(从物的)“行星的数量”这个对象必然是奇数;(从言的)行星的数量必然是奇数。当我们认为太阳系有九颗行星时,前者是真的,而后者非真。};另一类区分是按照时态模态\footnote{一个例子是“张三现在很帅气,但他完全可能是另一个样子”,时态意义下是真的。}和反事实模态\footnote{一个例子是“若二战未曾打响过,那么世界将完全呈现出不同的样子”,反事实意义下是真的。},前者侧重于描述时态意义上的可能性,后者则侧重思考取消某一事实之后的可能结果;也可以区分认识论和形而上学,分为认识论模态\footnote{例如“桌上大概有三个苹果,但是我其实看不太清”。}和形而上学模态\footnote{例如“三个苹果是否必然在桌上?它们为什么不可以在地上?”},前者的模态产生是由于其当下的认知状况产生的不确定性,而实际上的不确定性并不存在;而后者的模态产生于某种脱离认知而依赖事物本质的可能性,是就事物本身而非认知者而言的,描述的不确定性偏于实际。

关于一般反事实条件句同模态之间的关系,一般地认为二者存在一种逻辑上的联系。这种联系被这样地刻画:若A必然,那么非A将反事实地蕴含矛盾;若A可能,那么并非必有“A将反事实地蕴含矛盾”。
\subsection{可能世界及其语义学}
通常地,举例来说,有某一物体在特定的位置上,那么它就在这里——但是它是否必然地在这里?为了解决模态情形下的语义学和必然等概念刻画,我们提出模态命题逻辑的模型(MPL)。在语义敏感词那里,我们已经初步地涉及了可能世界与真的关系。这里的语义学的目的是能够系统清晰地刻画模态意义下的真值和命题等的问题。具体而言,模态命题逻辑模型M可以认为包含(W,R,$\mathscr{F}$)三个对象。这里,W代表可能世界的集合;R代表可及关系\footnote{亦即一种对可能世界之间关系的刻画。直观而言,从A世界到B世界的可及关系存在时,我们从A世界来看就得到B世界的各种情形都可能发生,换言之事实上就刻画世界的相似关系。举例来讲(取出一种特殊的可及关系),对于和我们的世界有同样的物理规律的世界被认为是可及的。像是“Kripke可以无外力帮助地飞向月球”,它必须在和我们的物理定律有某种关系的情形下考虑才能得到可能与否,否则其可能性在没有任何限制的情形下将成为无意义的。借此可以具体地描述“必然性”:若对于$\mathrm{W_1}$世界某命题必然。那么就说明在和$\mathrm{W_1}$可及的全部世界上这个语句都是真的。其他的刻画也是类似的。};$\mathscr{F}$代表解释函数\footnote{它把全部的由某一个原子命题或者语句同可能世界形成的二元组(P,W)指向到一个真值,亦即,真或者假;换言之对所有的可能世界中的所有的语句都给出一个真或者假以满足上述排中律条件,当然也就说明任何的命题都要在所有的可能世界中分别考虑真性。最后,还给出某一个确定某规则的函数,它递归地给出含有命题连接词的复合语句的意义。}。

作为语义学而言分析到这里就已经足够了。但是站在形而上学的视角上,为了进一步解释这里的元语言以面对更深层的哲学问题,对于可能世界本身,我们也给出一个形而上学的解释。我们的共识是,所谓可能世界,无论具体是什么,都必然具有最大化的特征,换言之,具备总体性的特征(以使得其成为世界而非仅仅包括有限的事实,对于各种描述性的命题都存在一个真值)。语义上升的说法是,任意的描述性命题在这个可能世界下都满足排中律;直观上即任意的事态都或者发生,或者不发生,这是可能世界的最大性所保证的。

第一个形而上学解释方案是具体主义的方案。其观点是,可能世界并不抽象,而是一个具体(Concrete)物,某种程度上是一种和我们的实际世界类同的平行世界,它们分别存在且分别具体,与我们的世界具有一种粗糙的相似性。这就取消了我们的世界的特殊性:它指定我们的世界是若干可能世界的一个,而实际世界就具备指称性,它表示说话者所在的那个特殊的可能世界。从而在这个意义上,可能存在的对象和实际世界的对象也没有什么形而上学意义上的不同,存在方式也是完全类似的;我们在某个特定的世界能够关注的全部命题只不过是一个世界与我们的世界的“临近性”。这就有具体主义的口号:“全部的对象都是存在的,所有其他的可能世界也都是确实存在的;只是不存在于我们的世界罢了”。这种看法就导致任意的奇特的对象都可能确实地存在于某个可能的世界,阻止某可能对象存在的阻滞性因素在具体主义的视角下根本不存在。不存在的那种对象只能是“圆的方形”那种分析上不自洽的对象。

对于模态概念的解释包含非还原论和还原论两个视角。还原论希望我们不引入具体模态的概念来解释可能世界,而非还原论认为不必把可能世界还原成一般性的概念,而能够接受使用其他的可能世界和模态概念去看似循环地阐释模态。具体主义进路至少解决了这样的问题,亦即它在解释模态概念的过程当中仅仅利用了较为通常的世界的概念,且我们充分利用了我们的实际世界的某种相似性。这就走向了还原论,它没有普遍地使用本质上为新的对象来解释模态命题。所以,这个方案给出了一个模态命题之所以为真的一个直观的解释,这一解释基于还原论视角是直接的。同时,它给予模态真理以非常保守的真值条件的解释,具体来讲,它只不过把非模态真理在现实世界的真是条件向其他没有本质去别的可能世界做了平移而已;可能性的模态真理也不过是偶然地与说话者所在的世界不同的某个并行世界的真理,其使真机制是类似的。这就使得我们也不需要给出一个对于可能世界而言不同于我们的世界的本体论,因为其本体本质上也平行于我们的世界。总结地说,它根本没有什么形而上地区别于我们的实际世界而存在的特殊概念。

但是,这种方式的问题也是非常明显的。无穷无尽的存在物和可能世界的膨胀,拒斥任何平行实体的否定存在,所有奇特的对象都是实在于某个可能世界的;进一步地,它对于从物的模态缺乏一种特殊的说明处理。具体而言,对于不同并行世界的同一个对象的同一性如何将成为一个棘手的问题,我们很难保证物体在跨世界的稳定性和辨识性。或者,在平行世界的对象能否认为和我们的特定对象是同一的;这就进一步使得,本质属性也成为难以解释的问题。更进一步的反驳是,作为可能世界的整体性而言,众多可能世界的世界集是否包含在某个更大的世界的一部分,如若不然,它是否有什么东西来保障这些小平行世界的独立存在?当这种保障不能被充分地解释,可能世界这个概念将因为不完备而失去作用\footnote{具体主义的回答是给出了一个把可能世界进行隔绝的条件(包含因果隔绝和时空隔绝等),这使得可能世界在定义上就不能结合;这样,他就把可能世界通过补充定义的办法强行分开了。D.Lewis的解释是,可能世界是时空关系上独立的整体。然而这一修补的定义也被指出是不足以充分解释的,且还会引出新的问题,这里只不过是通过增补规定来提高自洽性罢了。}。即使像是具体主义者声称的那样,可能世界之间的隔绝是真实存在的,那就涌现出新的问题(即孤立性问题):如果这些可能世界与我们世界的那种意义上的关系完全不存在,它们如何提供我们世界的模态性质的使真基础\footnote{举例而言,假如某人欲要成为北大校长,他将说“若我去年如何如何,我就已经当上北大校长了”。然而,在那些不存在北大,乃至不存在想当校长的人的世界当中,我们考虑模态的时候为什么要考虑这种意义上完全无关的内容?D.Lewis的回答是Counterpart Theory(对应物理论),我们考虑某个这意义上的词句时那一词句的“相似性(Similar)”是语境敏感的。当我们考虑 上述那种命题的时候,我们其实是考虑设想出来的和我足够相似的那种世界的情形。如果在所有这种情形下事情总是发生,那么我们就可以有充分的理由称之为必然。不过这一理论就抛却了那种字面意义上的必然性,而走向了一种更强的实在论(Robust),因为它只考虑具体的情形。这里,最靠近意指于其他任何可能世界相比都更靠近。具体如何定义最性不是我们所关心的。}?从知识论的意义上,这一理论的困难在于,为了获得知识所必要的“辩护”,我们怎么从这样的模态中获得那种辩护?这里就必须作一个阐释。

第二个方案是抽象主义的方案。其信条是可能世界的抽象性。至于具体到它的抽象对象如何,存在两种抽象的定义\footnote{D.Lewis对这一点当然是反对的。他认为抽象主义的观点是“语言的人造编织物”。}。第一个版本指出,可能世界是最大可能的事态\footnote{事态在Wittgenstein的《逻辑哲学论》里被如下定义:世界是一切发生的事情,而
发生的事情,即事实,就是诸事态的存在。事实和事态,在平常使用的时候,几乎并不加以区分。但是维特根斯坦特别区分了这两个概念。事实是事态的存在,也就是说,事实是所存在的事态。比如说,苏格拉底是柏拉图的老师这句话是一个事态,比如说,苏格拉底是柏拉图的老师,也可能不是柏拉图的老师。换言之,事实是存在着的事态。所以说,维特根斯坦区分了现实的世界和所思的世界,现实的世界是事实所构成的,而所思的世界是由事态所构成的。而这种所思的可能性,是指逻辑空间的可能性。可能的东西未必会成为现实,但是不可能的东西,是绝不可能成为现实的。因此,事实世界仅仅是逻辑空间的一部分。因此,对象=事物是所思世界的实体,事态是构成所思世界的最基本单位。而事态之间是相互独立,无法从一个事态的存在与否推出另一个事态的存在与否。同样的,事实是现实世界的基本组成单位。而事态的结构决定了事实的结构,也就是逻辑空间(所谓逻辑空间是表音文字体系特有的现象)决定了其发生与否的可能性范围,展现出先验主义的特点。};事态被定义成一个抽象的对象,它以不同的方式刻画世界。一个事态或者成立,或者根本不存在;而某一个可能的事态其实是表述一个可能存在(Obtain)的事态。特别地,当一个事态成立的时候,这事态就成为“事实”;而一个事实才是确实地可以为命题赋予真值的对象。这里,事态则只是一个抽象且不落地的对象。所谓最大的事态S,是指对于任意其他的事态S',这个事态或者包含它,或者不包含它。如果S包含了S',这当且仅当S与S'的存在性是统一的;若它不包含S',那么S与S'的存在性必然是相反的。这就解决了可能世界的排中律问题,它不允许某个事态的存在性(亦即,这一事态是否成为事实)在可能世界这个事态本身内部不能被判断。第二个版本指出可能世界是命题的最大一致集(或最大可能的命题),它尽可能多地包含了所有相容的命题(亦即,A或非A只能成立一个,且所有A-非A对都包含且仅包含了一个。全体这种命题构成可能世界集)。

然而这种解释也产生一些新的问题。最显著的问题是不能解释空名问题(尤其当某些命题的要素要求处理实在对象时):具体主义的那种较强的实在论将能够解释空名问题,它指出空名所指代的对象只是不存在于我们的世界,它总是存在于某个特定的可能世界当中的。其二是最直观的,亦即我们这个看起来很具体的世界作为可能世界的一员,其的抽象性是很难被接受的。

在抽象主义对可能世界的阐释当中,有多种形而上学的解释。第一种是纯抽象对象解释,即仍然坚持可能世界都是抽象对象,甚至于包括我们的实际世界;它承认全部这些对象都是永恒存在的抽象的实体。这里,抽象对象的实在论相比Lewis那种强实在论而言是较弱的,尽管每个可能世界都存在,但是它们并非全部成立,而仅仅作为某种实体存在,就像命题存在而未必真一样。当然,它没有解决上述反直觉问题;一般地,我们居住的抽象世界当然应该是一个具备具体存在物的世界,而非什么永恒存在的抽象世界。

另一种方案提出我们的世界的特殊性,指出可能世界事实上是存在者的实际世界的可能情形的抽象表达;这一观点被应用到Kripke的解释当中。其主要的观点是,仅仅承认我们的实际世界是一具体世界。所有能够刻画的具体对象存在且仅仅存在于我们的世界当中(具有现实主义的色彩\footnote{这就表明一切具体的对象都是存在的。相比之下,如果我们在承认这些实际对象的同时,也承认有些对象尽管不在我们这个世界实际存在,却作为某种可能存在的客体存在于其他可能世界这种观点,这被称之为可能主义(Possibilism)。})。自然而然,可能世界和命题等作为抽象物也实际存在于我们这个世界,它们至多是我们这个世界可能状况的一种抽象表达。在现实主义的观点之下,如果我们说某一对象存在于可能世界,事实上我们仅仅承认一个可能世界被以这种方式刻画时,这个对象仍然存在,而当这可能世界被否定了之后这个对象的存在性也就随之被否定了。语义上升地讲,包含这个命题的最大一致集的真值给定之后,“这个对象存在”作为一个真理是与之相容的。可能世界从而成为一个可能状况的真值的表述的总和。

此后的一些解释减少了可能世界的实在性。一种缩减主义的观点指出,可能世界至多是我们用以重新表达日常的模态语句的一种方式,从而事实上成为一个工具性的概念,不必深入地挖掘可能世界的形而上学意义;这种把可能世界作为工具的办法的确取消了之前那些棘手的跨世界稳定性和可能世界的形而上学本质之类的问题,但是它在取消了上述实在性之后面临的问题是:假如仅仅把可能世界作为工具,那么对于某个可能世界里的对象,我们如何解释“存在一些可能世界使得某对象具备某种性质”这种命题的可能真值\footnote{缩减主义的看法是,这种命题只不过是对“某对象可能具备某种性质”的阐释而已。}?进一步地,可能世界的引入与我们在日常语言的可能性事实上并不是很好地相容的,我们的日常语言已经足以解决关于可能性的问题。假定这里的工具性概念给出之后相较于日常概念更为复杂棘手,那么何必去拿出日常世界这样一个概念呢?

对于这一问题,另一版本的缩减主义补充道,可能世界的概念不是用于给日常语言中的可能性进行解释的:它是一个助发现(Hearistic)的手段。可能世界是一种可能性的描绘,就像卡通画是对于人物的描绘一样;虽然卡通画不能为人物作彻底阐释,但是它仍然可以用于发现人物的典型特征。我们利用可能世界这个关于模态的“卡通画”揭示出模态性质,就像是用手指头辅助计算算术题,它甚至取消了可能世界的工具性。

更进一步的观点是虚构主义的。虚构主义认为,全部我们所谈到的可能世界某种程度上都是虚构。它认为,围绕着可能世界的诸多命题根本不存在本质为真的命题,字面意义上使用可能世界的命题都是假命题;因为根本不存在一个实际的东西满足可能世界的定义和解释。但是某种程度上,为了满足某种特定的需要,我们假装这些命题是对的,而全部关于可能世界的讨论不过是在进行“假装游戏”,以便于描述一些对象或者实体的特殊性质,就像是数学那样。

唯名论观点\footnote{唯名论否认共相具有客观实在性,认为共相后于事物,只有个别的感性事物才是真实的存在。各种概念只是从现象中抽象出来的陈述,不是客观存在的事物,只是一个主观的名称。}则某种程度上包括了一些虚构主义。它承认一些可能世界相关的命题是真的,但是它的成真机制是不同于实在论的。虚构主义和唯名论的区别是这样的:虚构主义认为我们所有的那些有争议的那种对象的描述都是假的,仅仅在一些命题态度意义上才有可能是真的;其为假的理由是没有实例。现代唯名论者坚持当使用这些句子的时候它可以是真的,当我们说这个性质如何如何,那种真性甚至满足T-模式,但是,满足了这一点的同时我们并不一定承认世界上有那个对象,其为真是因为某种程度上有关于我们的说话方式。当我们说某对象有性质不过是对它概括,只有特定的个体存在,我们不过是刻画,这种语句因为刻画而真,不是语句指涉的个体本身存在;Tarski真条件并不蕴含那些对象存在,这也并不蕴含实在论是真的;这是我们的语言决定的,是语义学所蕴含的真。

\subsection{跨世界的个体同一性问题}
假定这个对象在每个可能世界中都必然存在,来考虑一个对象的属性:对于对象本质属性而言,它具有跨世界稳定性,亦即对于每个它存在的世界这一对象都具有那种属性;当然,通过某些个别本质属性反过来推得对象本身是做不到的。为了通过本质属性回到对象来解决个体同一性问题,我们需要划分本质属性。本质属性可以分为两类,一类是被多种对象同时拥有的本质属性,例如理性可以为不同的人所共有;另一类是一个本质的“自性”。在“自性”的性质当中,有一些是非本质的(例如“22世纪出生的第一个人”,这个性质当然是自性,但不具备跨世界的稳定性);被哲学家建议并广泛讨论的本质自性包括与自身的同一性(换言之,“我”就是“我”,而别人不能成为“我”),某对象的起源或者建构\footnote{举例而言,对于前述的Thesues之船那里,建构的观点指出:全部的板被更换之后,船就不再是船了。亦即,一个对象的建构是它开始被建立起来的那种建构范式。}。
\section{使真者理论的基本观念与哲学问题}
这里讨论的是以使真者模型为基础的语义学和真理对实在的形而上学依赖。使真者语义学的口号是,“一个句子为真的原因是世界以何种方式使它为真”(A sentence is true because of how matters stand where its subject matter is concerned)。使真者理论的重要内容包括,世界使得命题或语句为真的那种方式(Way)和一个事实是因为它关于(Aboutness)的使得其为真的那个主题。
\subsection{使真者理论的产生动因}
使真者理论之所以是被需要的,来自于两个方面的需要:语义学的需要和真理解释的需要。对于语义学而言,已有的内涵理论和可能世界的刻画还远远不够细致;那种语义学的“格子”过于粗疏,对于具体的(例如内涵相同的语句之间的意义区别)语义刻画缺乏解释。这一点也是为哲学解释(例如刻画不同的必然命题之间的区别)所需要的,但是过去并没有能解决这一问题的工具。就真理解释而言,直观上真理依赖于实际的世界的存在物,但是世界却不是依赖于语句是否为真才能存在的,因而真理需要更本原的阐释(而不是天然成立的)。我们对真理成真的理由还缺乏系统阐述(尽管有些符合论者声称仅仅依赖符合事实就可以解决真理来源问题,但是还有反事实条件句的真性等问题亟待解决);并且还需要一个真理的“分辩者”,亦即寻求一个判断那些宣称为真理但实际上没有实在基础的断言的伪真理(Cheater)。

换一个角度说,语义学的使真者理论的目标是通过一个使真理的承担者(A bearer of
truth)为真的被引入的假定存在的要素作为模型的基本构件来给出语句的定义,依据这个更细致的构件(而不是粗糙的内涵或者指称等)给出语句的实际意义;形而上学的目标则是试图直接获得何种存在的实体使得语句成为真的,当每一个人的直觉都强烈地指向语句确实能够断定关于实在的某种东西这一想法。

我们来举出一些具体的例子。伦理学的例子是这样的:假定通常意义下“A帮助一个有困难的老年人,那么A应该得到赞扬”为真,而“A帮助一个有困难的老年人”与“A或者帮助一个犯罪上有困难的老年人,或者帮助一个正常意义下有困难的老年人”在任何可能世界当中都必然是等价的,那么就有“A或者帮助一个犯罪上有困难的老年人,或者帮助一个正常意义下有困难的老年人是值得表扬的”,这自然是反直觉的,就需要刻画这种同义替换下的相异。在真理论上,一种传统的说法是“内涵等价”:Lewis指出,命题的内涵是所有使得它在其中为真的可能世界的集合,而内涵等价就是两个命题在为真的可能世界的意义下完全一样。这里的例子当然符合内涵等价——更一般地,在可能世界语义学里,所有的必然命题都是内涵等价的,例如那些数学命题。那么,这种解释甚至不能区别必然命题(直观的例子,两个不同的数学真命题)的相异,但是在这个伦理学例子里我们看到这种区分的必需性。第二个(因果意义下)的例子是,“若A推球,球将向前滚动”直观为真,但根据上面的内涵等价,通过可能世界允许的替换就能得到“若丧失体力的A或正常的A推球,球将向前滚动”这种不符合因果律的命题。第三个(关于必然性等模态的)例子是“9为奇”和“太阳系行星数目为奇”的内涵等价;还有心理态度的例子等。从这里看到,必然等价的命题似乎在直觉上\footnote{逻辑形式是,关于命题$p$,$p\lor(p\land q)$和$p\land(p \lor q)$真值相同却意义不同。}有不同的意义。在使真者语义学的角度上,可以通过命题关于某个特定主题真和世界使得命题为真的方式这两个内容来考虑这里的区别。这种内涵以上的区别某种程度上就是{\heiti 超内涵}的区分。

形而上学的需要的例子是这样的(它指出我们需要使真者的形而上学):分裂与融合观点的意义上,我们用何种工具阐释“组成桌子的电子和电子的位置关系”和“桌子”两个对象的独立性?我们在这里至少给出一种工具来回答二者是否是一致的。对于更复杂的社会实体来讲,举例而言,一些有资格的人在适当的时间或者地点做了一些物理标记就宣称一个法律诞生,那么在这些物理实在之外是否存在什么别的东西使得法律的产生为真?另一个例子来自Ryle:他指出民间心理学声称的心灵实体都可以使用行为倾向(而非实在论者所说明的实体,实体性依赖反事实条件句的使真者的存在性)来解释,从而没有“Ghost in the machine”;那么又是什么使得这个命题态度句子成为真的呢?
\subsection{使真者语义学}
使真者语义学的轮廓像这样被Kit Fine具体地刻画和定位:真值条件语义学形式上分为条项式(某种程度上像是Davidson那种语义学,通过若干条项的句子及其逻辑关联而非依赖对象或者世界给定语义)的和对象式的,而对象依赖式的语义学就被联系的角度看又有走向可能世界的角度(内涵仅仅细化到可能世界的程度而不具体考察深入到可能世界内部的结构)和状态式的角度(借助Truth-maker作为模型给出超内涵的语义学系统,是深入可能世界内部的概念,像是情形或者事态的模块)\footnote{通过更细分的这种模式,像脚注113那里的那种命题$p\lor(p\land q)$和$p\land(p \lor q)$尽管真值相同,也可以通过具体使真者是关于$p$亦或者关于$q$来研究这里的区别。};状态模式的角度更详细地在关系上有松散的、不准确的和准确的三种分支,而使真者语义学最终归属到关系明确(Exact)的那个位置\footnote{何为关系的松散性(Loose)?举例来讲,对于之前的三个苹果的例子,单调性问题是指,若我们讨论了$\mathrm{A_1,A_2,A_3}$这三个对象的问题,我们是否同时描述
了任何其中的组合,它们都是关于它的问题?当我们单独地谈论$\mathrm{A_1,A_2,A_3}$,其中任意的组合是否也是它所关于的问题,亦即是否意味着我们谈的是关于三个对象的整体、乃至于包含这三个对象的更大的整体的问题?在使真者语义学的方面,我们必须阻止这种语义学包含的对象的不断扩大(乃至基本结构的关系松散到可能世界大小的程度;这种语义学上的对象之间的包含关系问题事实上满足偏序关系的性质),所以我们拒斥松散的关系而必须限制类似单调性那种性质。关系的准确性要求一些细致而准确的关于的刻画,并且通过这种方法框定具体的细致对象。在使真者语义学那里,假设一个A是命题P的使真者,不能加上一个无关的B,并称A和B的fusion是p的使真者。因为直觉上,使真者必须是和命题的真完全相关的,使真者里不能包含和命题的真无关的部分。这就是关系对基本结构的限制。}。从另外一个角度来说,对比之前的语义学观点,使真者语义学是对命题的不同刻画\footnote{命题在不同的语义学观点下有不同的看法。Lewis内涵下命题是使得其为真的可能世界的集合;使真者语义学把基本结构细致化到“命题是确证者与否证者(而非成真或者不成真的可能世界这种粗疏的结构)集的二元组”。这与经典的Russell结构性命题指出的由组成部分“组合”而成的复杂性实体的极为细致的概念也不一样;某种程度上使真者语义学介于二者之间。}。

从这一刻画出发,我们来考察更抽象的使真者语义学的概念。其的关键想法是使用更小的“事态”、“事实”亦或者确证者这种结构以取代可能世界对语义的刻画。在Kit Fine那里,他并不从形而上学的角度来关注可能世界及其释义的价值,而更加关注可能世界的理论对于语义学的阐释。换言之,语义学视角对可能世界的形而上本质、存在性之类并不关心(倘若这些内容无关于我们需要的语义学的范畴),它只关心可能世界究竟何种程度上对语义学而言是“有用的”,并把它加入到我们关心的本体论当中。这就是使真者语义学考虑问题的方式。既然具有这种工具导向,那么类似地,其使用的具体的技术对象,都仅仅是一些松散的抽象概念,用于刻画一些语义学驱动的性质;我们不必在这里彻底探究这些哲学对象的形而上学本质。

回到开始的“Way”和“Aboutness”那里,事实上使真的“Way”亦即以使真者作为模型的建构基础,在典型的情况之下只是可能世界的某一片段,而这一片段在形而上学的意义下可以是事态、事实、状况或者事件等;“Aboutness”则反过来,使得使真者概念给了语义学一种可能,从语言本身的结构和内容方面它可以至少看起来是同世界中的情形相关的。传统上,真值条件是一系列不直接涉及关于性的条件描述(像戴维森纲领那样,是若干真的具象之和),或者是笼统地关于作为逻辑点的相互关联的可能世界的。

对于Kit Fine而言,他要求使真者是“具体而确切”的,而不希望包含任何同使真无关的对象。他阐释道,某个事态如果是一个使真承受者的使真者,那么所有包含它的事态都可以是使得它为真的那个对象,但是语义学上的使真者必须施加一个限制,使得其不能在范畴上包含太多的对象。其模型是这样的:首先考虑一个状态空间$(S,\subset)$和更具体的模型(一个有序三元组)$(S,\subset,v)$。这里和可能世界语义学的定义很相似:$S$是状态的非空集合,$\subset$则是$S$上的一个偏序关系。这里,状态空间的所有子集$T$(可以看成是某种使真者,它被这些事态的集合蕴含)都至少需要有一个最小的上界$\cup T$作为这个子集的元素的分体融合(Mereological fusion)来控制子集的扩张,同时又存在最大的下界(作为基本单位,这样就使得这个空间成为一个格/Lattice)。这个函数$v$就是一个真值(Valuation)函数,这函数是从每一个命题到状态空间的两个子集(事实上就是使真的子集和使假的子集)。

这一模型产生了诸多新的结果,并丰富了语义学的可能性。第一个最重要的结果是,通过把可能世界细化为状态空间的子集,我们能在超越内涵的意义上为命题做语义学的区分(换言之,超内涵的区分)。这里,与可能世界的内涵理论不同的一点是,可能世界的最大性保证了对每个命题而言,每个可能世界都或者使之真、或者使之假(尤其在所有相容命题的最大一致集那个定义上可以看出这一点),但是事态的子集不能保证这一点;所以在模型那里我们已经说过,必须指定一对子集:使真子集和使假子集,才能够确实地决定语义学的内涵,而且某种程度上正是通过这种决定子集的方式才使得我们能够更细致地刻画命题的语义学。

另一个重要的结果是这使得我们的语句(命题)的部分真性是可以刻画的,我们对于一个整体性的语句$S$,如果它由$n$个子语句构成,其中一些为真、另一些为假,那么我们就可以合理地通过这种概念获取“部分真”,虽然这个语句$S$本身是假的\footnote{来自Kit Fine的例子是,科学的上升性;在可能世界的角度上,我们的科学的真性总不能被良好地刻画,但是藉由这个部分真的概念,我们就可以考虑一个逼真度;科学的进步正是使得语句为真的比例上升的过程,利用这种想法就可以确证科学发展在语义学上的意义。这是科学哲学层面上的一个应用。}。而可能世界只能说明其在哪些可能世界上为真,从而使部分真的刻画只能在可能世界的层次上而非语言内容本身这一层次上面被讨论。

关于使真者和关于性的概念,在这个模型下仅仅形成相互说明的关系;使真者是语句所关于的,语句所关于的也是使真者;但是那并非一个一一对应,换言之,使真者并不等同于所有这个语句所关于的对象。真理的刻画不再是空洞的真,而是“$S$关于$Y$为真”($Y$就是语句所关于的,换言之,主题);特殊地,关于性这一性质也是超内涵的,内涵等价的语句同样在关于性上也不相同,允许关于不同的对象。S.Yablo说,这就使得真从简单的真值上升成为“关于某个主题成真”。

另外一个直接的结果是不可能状态被自然地引入了语义学考虑。与那种加入了不可能世界的可能主义语义学相比,它仍然是做着语义细化的工作。这种“不可能的”事态的特点是,内在地蕴含了不可能或者矛盾,但是它仍然是有意义的使真事态。不可能事态的引入能够区别不同命题的不可能状态\footnote{例如,对于“A是彩色的”,它的使真者包含“A是红色的”、“A是蓝色的”等等;通过定义不可能的事态,就可以刻画这样一族子命题共同构成的使真者,它们不能够同时成立,但是可以通过研究这个状态细致地描绘诸多事态的Fusion,换言之,它在这里是技术性的。},尽管其内蕴矛盾但是相比不可能世界的描述更详细;更技术性地讲,它能够处理必然假命题之间的区别,使得“A是圆的和方的”同“A地既下雨又放晴”不同,尽管在可能世界的意义下是一致的;因为它们指涉的不可能状态并不相同。另一个意义是用来考虑反可能前件的反事实情形\footnote{例如“若A发现大于2的偶质数,他将获得菲尔兹奖”和“若A的体重大于100公斤且小于50公斤,他将获得菲尔兹奖”在所有可能世界下都是严格反事实的(然而是真的,因为条件永不可能被满足),因而难以区别开来;但是考察不可能状态的差异之后,自然而然地这两个命题的前件就在语义学上区别开来,意义在于发现了它们是关于何物为真的。},通过对不可能状态的不同来考察这些反事实命题的不同,进而语义细化地区分之。总地来说,某种程度上不可能状态之间存在的差异就是其细致化的本质要素。利用不可能状态Fusion于不可能世界当中的特点,我们才能刻画那种使真者的集合。
\subsection{形而上学的使真者理论}
形而上学的使真者和语义学的使真者相比具备某些不同的特点。语义学的使真者某种程度上是平行的,其所处的位置在真理和事态之间,而真理的内涵具体地在这里是被那个“使真事态集”(而不是可能世界)所刻画的。这里,使真者理论所研究的问题则更趋于本质。形而上学的使真者居于真理和世界上的实际对象之间,其真性是由使真者的存在性(而非像事态集观点那里的子命题真性)刻画的。对于语义学来说,只需要关心使真者去保障语句的真这一问题即可,但是作为某种形而上学的使真者,仅仅关心这一点当然是不够的;真理的形而上学本质是跨范畴的,至少,我们需要考虑这使真者的非技术性特点。

目前这一理论关注的主要命题有这样的几个:何种语句(或真理承担者)是具备使真者的\footnote{一般的观点是最大主义(Maximalism):如果p真,那么至少存在一个实体$\alpha$使得这些实体保证p的真性;也就是说使真者是最大的广泛概念,所有真都有使真者,这蕴含着某些像可能世界那样的最大性。有观点指出,有些使真承受者根本没有什么使真者,这被认为是使得我们失去驱动使真者理论完善的动力的。}?使真者本身又可以是何种对象\footnote{因为与语义学的那种粗疏的技术性定义不同,这也是亟待讨论的。主要的观点是事态和事实等,但是关于事实本身也需要进行讨论。}?真理承担者与使真者的关系又如何\footnote{一般的观点是,$S$的使真者存在时,$S$真。这被称为是必然主义观点。这里的争议是,这种强的约束未必是我们所需要的;有论者并不承认Quine的本体论标准,从而否认使真者,只承认真是某种世界上的存在为之负责的。},使真关系具备何种关系性质\footnote{像是讨论两个使真者之间的使真关系自反、传递、封闭、对称等性质。}?就本体论承诺的角度而言,我们做出本体论承诺的时候是否同时承诺了那些使真者\footnote{主要的本体论承诺讨论问题有二:应当遵守Quine的量词承诺标准还是使真者承诺的成真标准,以及使真者理论是否承诺了关系?主流意见对第一个问题的回答指出使真者的真承诺才是标准,而第二个问题则被广泛地持肯定态度,使真关系自然蕴含承诺实体。然而,有论者指出,不必存在实体也能解释使真关系,它只被用作有意义的某种解释。};进一步地,使真者理论是否是必要的,真理需要一个具备独立理论意义的使真者理论吗\footnote{使真者理论的持论者(像是Armstrong)当然肯定其必要性,但是素朴实在论和早期的一些强实在论认为只要承认对象的实在性,加上一些符合论的真判断就可以解释真理了,使真者理论是多余的人造物。其他的观点也包括指出形而上使真关系只不过是给出了一个关系着的二元组罢了,至于意义是空洞的(Kit Fine);寻找使真者的过程已经偏离了使真者理论的初衷;使真解释事实上还依赖于更本质的解释,甚至用ground(使得某物有根据的理念)这种最还原的对象就够了,使真实体或理论根本就是无谓的。}?更根本地,使真者理论本身是一种什么样的理论或者概念\footnote{需要澄清的观点主要包括区别使真者(非对称)和符合论(对称)、本体论依赖、使真者的使真过程依赖于何物等。},它同其他的形而上学的解释(例如Grounding Theory)\footnote{形而上学家们一般认为ground是一个不可被还原的概念,所以只能用一些典型的、大家具有共识的例子来刻画它: That the apple is red grounds that the apple is colored; That the apple is red grounds that something is red; That the apple is red grounds that the apple is red and the apple is round,但在很多情况下大家对于what grounds what是有争议的,这些争议构成了当代形而上学讨论的核心。Kit Fine认为只需要一些本质性的本体和“Grounding”,对于形而上学就够了;P.Audi则更进一步,指出根本就不需要使真关系那么宏大的关系。为了阐释使真关系,又需要若干还原的小的观念,像是Grounding或者因果;而当我们清楚地了解了全部这些关系,本来的使真关系就不需要了。当然也有人认为Grounding理论因为过于还原,所以不被需要。}有着何种关系?
\subsection{否定存在陈述的使真者}
D.MUÑOZ这样说道:“人不能够生活在否定之上,但是脱离否定使得我们无法生存”。其实,这阐述的是这样的问题:否定存在问题看起来是符合直觉的直观真理,但是它的解释对所有的形而上(当然,包括使真者)提出了挑战。传统的否定存在问题我们已经在语言哲学那里讨论过;它描述的是一些纯粹的虚构对象或者非一般意义上的某种神话化的抽象对象,像是“A不存在”这种语句或者命题。这种否定存在的陈述在直觉上是真命题,但是它对于我们对语句真的条件的传统解释是冲突的;因为简单的语句有意义基于我们相信指称位置的词指称一个实际存在着的对象,这个对象或者有、或者没有被谓语描述的性质,而真则说明那个对象具有着被描述的那个性质。严格来讲,我们在考虑否定存在问题时承认以下前设:指称词被使用这件事本身就蕴含存在,语句有意义在于被指称的对象具有或不具有谓词所述的性质;某对象只要被称之为对象,它就蕴含着存在;而性质只能蕴含于存在的对象内部。

否定存在面临的难题从而就出现了:对所有否定陈述一般性的问题是,当上述前设未被满足的情况发生,语句的真假如何判断,又如何能够有意义(即,所谓空名问题)?对于空名的特殊情形,即否定存在问题,特殊的问题是:我们说一个存在的对象不存在根本上是蕴含矛盾的,但奇怪的是没有什么人直观地觉得这有问题。形而上地讲,一个不存在的对象根本上为什么会有性质(即“存在性”)?

对于这些问题,在历史上出现过许多关于这些问题的独立于形而上学的解决方案:首先是梅农主义的解决方案,它直接试图否定之前我们提出的那种前设,而提出了一套完全独立的语言学前设\footnote{简单来说包含两点:无关性原则,即存在与指称对于一个对象来说根本上是分离的;独立性原则,即存在本身是特殊的,根本不是什么性质。这就解决了上述的问题,但是由于其规定性不被广泛承认。}。我们熟悉的Russell的解决方案就是否认日常用词的指称词,而是把所有词汇都认为是伪装的摹状词,而摹状词根本就不是什么独立的语言学对象,它还可以被进一步还原,具体的方法和问题我们已经在摹状词那里阐释得足够清楚。最后一种解决策略则是否定了Quine的本体论承诺\footnote{即量词与名字的语言学承诺。其标准如下(这是一个重要的标准,值得记住):{\heiti 经过规整化的真语句中的量词所约束的对象,就是语句所在理论本体论上承诺的对象。}如果我们分离量词和名字,那么我们在使用量词的时候根本就没有做什么本体论承诺,自然也就没有什么否定存在难题。}。这些解决策略都或多或少地遭受反驳,而且是独立于使真者理论的。

在使真者理论出现之后,否定存在理论基于使真者产生了若干新问题,使真者理论的坚持者们必须要为否定存在难题在使真方面找到一个解释。否定陈述在使真者理论的一般性的问题是,否定陈述的使真者是模糊的(尽管使假者存在,但是对于否定语句的使真者和语句本身的使假者而言,二者并不能直接统一起来,某种程度上是因为使真者不具备最大性)\footnote{使真者理论的典型解释有二:为否认A是红的,第一种观点是利用整个世界作为整体事实去排除A的红性质,利用世界来框定非红;另一种是承认否定事实,即“A是黄的”是蕴含不相容性质的事实,但是引入否定的事实判据似乎总不被人们的直觉所同意。后面还将详述这里的困难和解决方案。};就否定存在陈述这一特殊问题而言,用Aboutness的视角来看,既然我们称“A不存在”,那么这个命题至少是关于A的;而若一个句子是关于A的,它当然至少要某种程度上存在。然而,一个不存在的对象是如何行使使真者的职责的;如果它不是使真者或者关于的对象,那么A的成分又如何?对于使真者语义学来说,需要解释的问题就是这一语义学下否定存在句的逻辑形式和语义表达模式是如何的?

在使真者理论的视角下,一些对于这些问题的解决方案得到了提出:最素朴的解决方案是认为否定存在语句根本没有什么使真者,事实上就是否认了我们在形而上学使真那里提到的最大主义,即所有真对象;但是这一理论并不被广泛支持,根本原因在于反对最大主义的观点就是不受欢迎的。假定我们去否认最大主义,就必须去解决这样的问题:什么样的对象有使真者,什么对象又是不言自明、不需要使真者的?这个界限问题就是最关键的问题,假定我们直接指定否定存在不需要使真者,那么这个概念就成了很技术性的概念,离我们提出使真者理论的初心也就渐行渐远了,使真者也失去了存在的意义(例如给出Cheater(假真理)的判据)。没有使真者的观点的坚持者Molnar提出了一套关于世界的理论,以此解释否认最大主义之后造成的新问题。其理论框架包含四条:“世界是全部存在物的总和”;“所有存在的对象都是正面的”;“部分负面的关于世界的陈述也是真的”;“全部的真陈述都是被一些确实存在着的东西保障的”。这里尽管某种程度上保证了使真者理论,但是确实解释了否定存在的情况下没有使真者而真的理由,因为没有什么存在着的东西为其提供保障。然而这一理论也仅仅是自洽,并没有得到公认。

承认否定存在具备使真者的一种解决方案则是解释有使真者,但否定了必然化原则(即我们之前提到过的,S为A的使真者当且仅当S存在时A必然地为真);假如抛弃这个原则,我们其实可以通过考虑某些实际性原则缩减使真者(因为使真者的要求被放宽了),使得使真者只包括一阶事实。Colin等人指出存在“负面真的正面事实”,以期通过放弃保障真的方式让一些否定性事实成为使真者,也能够刻画否定存在语句的内涵;某种程度上,就是让肯定存在语句的使假者成为否定存在语句的“使真者”(尽管不保证真而只是刻画语句内涵)。例如“缺乏”这词语本身就可以成为一种否定性事实\footnote{例如“理教207没有河马”,河马的缺乏性本身就是那个使真的一阶事实,具备Essence的性质。}。其拒绝把命题赋予事态,只承认事态依赖于其关系的命题,而不是命题本身“蕴含”于其中。这被提出了一些反驳意见\footnote{J.Parson指出,存在小的否定存在(如理教207里没有独角兽)和大的否定存在(如没有独角兽)。如果像后者独角兽这个客体本身就被否认,仍然不能把一些基本的所谓“缺乏”事实拿来刻画某个否定存在。},根本上,放弃形而上学的必然性将使得必然真变成偶然真,这并不为形而上学的观点所普遍容纳。J.Schaffer则认为,整个世界是一个唯一且统一的基础使真者,只有它能够赋予真;确切地讲,对于全部的世界,W这个世界内的唯一统一基础使真者就是W本身,它为世界内的各种事实保证真值(当然,跨世界的真值稳定性就消失了)。

对于部分保留必然化原则的观点,否定事实的增加被认为是无必要的,即“节约方案”。Cameron的意见是,每个世界W是那些“对于那些使得否定存在为真意义上不可分辨的世界”里的使真者,这就保存了必然主义(即真的跨世界稳定性)和最大主义(每个事实都有一个特定的世界作为使真者)。另一个极端则是承认否定事实整体地作为使真者的必要部分,它们存在于一般普遍的情形下,我们把事实和否定存在事实一并作为某种意义下的时空中的实体;既然肯定是对象和性质之间的某种范式,否定当然也可以是。Armstrong的观点则更进一步,他指出否定事实被包含在某种总体事实当中,所有一般的一阶事实和二阶事态\footnote{举例来讲,若“红色”是一阶性质,而“颜色”是二阶性质。这里,“A性质未被满足”本身是一个超越A性质的层级而存在的高阶复合性质,当我们说“没有对象满足A性质”,这本身就可以看作一个性质,就像Kant把存在当做描述性质的性质(亦即,二阶谓词)的思想。没有什么对象具备“存在”这一性质,只有“没有对象满足某性质”这种高阶性质,这是被包络在总体事实内部的。当我们谈及某对象不蕴含总体性质,那也就是不蕴含所有的某种性质,这一事实本身成为超越那些一阶性质的二阶性质,这样就实现了所谓的必然性。}都包含其中。这里,某种程度上蕴含了全称量词和特称量词的某种关系;所有事态都在这个总体事实内部的时候,自然就不存在什么不包含在这个总体事实里的了,否定事实也自然如此。这样,对于每个否定存在问题,我们都能在那个问题的限制下找到其中的使真事实。

\section{实在论与反实在论:问题和论证}
这里将涉及对于实在论和反实在论的一些论证。为了论证相关的事实,我们来首先考虑一些基本的实在论的内容。对一般的直观而言,容易接受的是素朴的实在论。其基本观点有二:真正的实体独立于我们的思想和语言存在,以及那些实体都存在于我们的世界(甚至,我们的世界就是这些实体构成的)。进一步的观点是主题实在论,它是一种以特定类型对象为主题的实在论,即承认某个或某类特定对象的存在;特别地,这种存在一定是独立自在,它独立于我们的心灵、思想和语言,不依赖于我们而独立地存在。素朴的实在论并不蕴含主题实在论;主题实在论实际上蕴含着一系列实在论,例如科学的主题实在论承认科学对象那样的实体的存在等等;而这未必为素朴的实在论所包括。反过来,素朴的实在论为每一种主题实在论所蕴含;每一种主题实在论的论证都某种程度上蕴含了素朴的实在论,反过来则是必然不可行的(除非那个证明本身就是某种程度上通过证明一个主题实在论产生的)。
\subsection{典型的实在论证明}
对于实在论的论证,针对那种素朴的实在论,由于它本身就是模糊的,作出对它的标准证明自然并不容易,因而仅仅是我们的素朴观念能够为其辩护。而针对主题实在论,一种论证是“不可或缺性论证”,其基本想法指出它要求确实存在的那一类实体对于支持某些具备实在根据的对象是不可或缺的;如若缺乏了这种独立实体的实在性,这种实际的对象(在直观上容易承认)却不能得到充分支持,借助这种类似归谬的办法来考虑其实在的必然存在性。第二种论证则是通过约束X主题的量词的使用的不可避免性和严肃性为基础的论证,它依赖于Quine的本体论承诺标准本身的正确性。我们在此再次陈述Quine的本体论承诺标准:{\heiti 经过规整化的真语句中的量词所约束的对象,就是语句所在理论本体论上承诺的对象。}假定我们承认这个标准,那么当我们去在真语句的规整化形式里面用了什么量词去约束那个主题下的对象X,那么X当然是存在的(依照本体论的承诺)\footnote{但是这里的反对意见是:用量词承诺的一切对象都某种程度上是被承诺的,似乎太过于宽泛了。当我们说“A has 2 problems”,则Problem就成为实在的;通过这种语言学的方式,似乎过于轻易地为客体赋予实在了。}。

语言学的论证是Moltmann提出的:它关心每个具体的关于某个抽象事实的名字的句法功能。既然那种名字具体地实现了某种语言上的功能,那么它就有必要作为一种实体存在。根本上,这里的论证必须承认意义的实现某种程度上依赖于语言;当语言的一种实体性质被建立起来,自然而然地,只要词句在语言学当中有意义,语言学论证当然就给予抽象对象的实在性\footnote{Hofweber的例子是:我们定义一个“A想与之约会的人”,当我们真的想到这个人的时候,它就被量词所限制,而且成为实在的。}。其他的一些论证包括“最好的解释”论证,即当我们为这些主题的对象X赋予实在性的时候对某个特定的Y是最佳的解释方式,那么就没有必要拒斥X的实在,因为那会引入更多的麻烦,同某种简单或者奥卡姆剃刀的原则是相冲突的;以及某种借助问题-回答范式产生的缩减主义论证\footnote{他们的解释是这样的:考虑“There are 3 apples on the table”,进而有一个客体“The number of apples on the table is 3”;那么,至少有一个数字“The number of apples”也作为某种属性存在,所以数字(至少自然数)是实在的。Yablo等人的反对意见是,从第一句到第二句的过程不是演绎的推理,而实际上是“放大的推理”,无外在条件的前提下似乎不能被推理得到。更激进的批评是,在使真者视角下,前两句话在Tarski的可能世界意义上内涵相同,但是假如使真者语义学的角度上真值条件也是不相等的。}。通过某种确实的实在物蕴含着一个抽象对象,它们在被考虑的时候当然就在那种实在物那里,所以抽象对象同它们的实在来源一样,作为某种实在性问题的回答时就被赋予了实在性。

\textbf{\kaishu 参考内容[13].}
{\kaishu {\heiti 语言学论证和缩减主义.}
语言学论证一般意义上讲是一个(或者至少相对于哲学而言)更偏重于经验的,或者偏重于操作层面的这么一个典型的论证方式,语言学的论证,某种程度上是作为消解或者说是判定某些实在论问题,或者说对某些语言所对应的对象是否存在这样的问题的一个判定操作,其实是没有特别强的先于判定之前的比较强的哲学的预设立场的。他的哲学,某种意义上讲是相对比较素朴的,也就是说语言学论证是设定了一系列语言学的,既有研究成果,在这个研究成果的基础上,希望考察我们关于语言的所谓的这种语言学意义上讲的科学的或者正确的性质、理解或者使用方式,那么,在这种正确和科学的这种理解的使用方式下,如果相关表达式是真正的有指称性的,那么,像Moltmann这些学者就会认为这样的一些表达式背后所他所指称的对象就是有某种或者至少可以有某种形而上学或者存在论意义上讲的承诺的。

那么从这个角度来看,它是一个判定性或者操作性的这么一个标准,也就是说你通过了语言学测试,那么你相关的表达式是真正的指称性表达式,这个表达式所对应的对象,我们是可以进行形而上学或者存在论或者本体论承诺的;当然没有通过相关语言学测试或者被语言学既有的研究成果证明是“误解”了相关语言的实际的、真实的结构和真实的使用特征的话,那么相关的表达式就并不是真正的指称表达式,在这种框架下,那么就不存在这些哲学家误以为这些表达式所指所能指称的一个对象,因此要带来的那个进一步的哲学的,或者是存在论本体论意义上的承诺,这是这个Moltmann这些学者所要做的工作。

事实上,这是一种比较典型的操作性的,或者是判定性的这么一个论证方案,因此,我们不太倾向于把他的解读赋予形而上的哲学背景(比如说他是否在这种意义,或者说这样的一些刻画下有比较强的哲学预设);那么缩减主义的观点是适合把这样的一个立场或者观点看作是比较典型的哲学立场。一般说来,那种所谓的紧缩论,它就是在刻画或者在主张在哲学层面上,不要赋予语言表达式背后过多的形而上学或者本体论的实质性的意义或者实质性的内容,在这种情况下,我们用一个语言去描述或者说刻画或者让我们来讨论某些内容的时候,这些内容并不见得因此就具有了形而上学的立场或者形而上学的地位。换句话说,语言学论证和缩减主义的观点的某种本质区别是,是否承认了这里的本体论承诺标准以及是否有较强的哲学预设(例如语言学测试对本体论承诺或者形而上的事实性质的保障)。

就Deflation Argument而言,我们不承诺这样的本体论,所以这倒是一个比较偏哲学立场的一个判定,因此,这是一个一般性的哲学的理论立场或者哲学上的一个理论性的这种主张或者方案,在这个意义上,它是预先设定了一系列的这个操作的,也就是说它或者是预设了一系列的这个哲学立场(比如说缩减的、或者,一个对象关于何种对象就会产生对这种对象的实在性相关刻画的这种紧缩的存在论)。

比较实质的表现是,我们到底要不要接受相关的,一般的哲学论立场;或者说你接受的是一个比较偏经验层面或者偏这个操作层面的而不蕴含特别多的哲学预先设定立场的这么一个操作。Moltmann,这些学者所做的语言学论证是偏后者,他不做预先的哲学判定,它只是一个操作性的,或者说一个测试性的这么一个立场。那么满足或者通过了相关操作测试的话,那么就有形式上的存在,否则就没有;缩减主义则是一个这样的一个立场,也是更强的一个哲学上的立场——他认为语言能够讨论谈论某些东西,并不意味着这些东西就因此具有了形式上的本体论地位和存在论的这样的一个地位,这是两种比较根本性的差异。

最后,就策略层面来看,语言学论证是类似于罗素分析限定性摹状词—罗素是通过澄清限定性摹状词的“真实的”形式语义结构,证明处于语句主词位置的限定性摹状词其实并不是referring表达式,而是denoting表达式,因此并不会由于出现了限定性摹状词就要求我们在本体论上承诺对应的对象。与此类似,语言学论证也是通过语言学研究成果澄清哲学家们误以为是指称功能的表达式,其实并不是真正的指称表达式,因此也不带来相应地本体论承诺要求的(当然,按照语言学理论成果,那些通过语言学测试的真正的指称表达式是存在着对应的本体论承诺要求的)。
}
\subsection{典型的反实在论证明}
尽管反实在论看起来要取消大量的实在性,但是我们需要在这里声明的一点是:部分反实在论者也承认素朴的实在论,故而有些情况下这些反对事实上只是对主题实在论的反对。一种关于语言意义的表达主义\footnote{简而言之,即语义学是研究语言和世界关系的学问,语言是表达世界的产物。}(Representationalism)观念一直相当流行,即我们的断言性思想(信念、断定、陈述,等等),及它们的语言承载者表达相对于这些思想的外在事物(指称关系是它的一种典型的方式),从而获得其意义。如果我们在表达主义的基础上加上实在论的前提,即这些外在事物被解释为独立于心灵的,就有了“表达问题”(Representation Problem):思想与那些据称独立于我们思想的外在事物,如何建立起表达与被表达的联系的。这些问题是表达主义加上标准实在论的产物,某种程度上也产生了怀疑论(因为独立于心灵实在的客体的对象是否能够透明地在不同说话者的思想内统一起来)和意义理论中的一些问题。通过放弃实在论、表达主义或者乃至两者,就可以解决很多哲学上滥觞于这两者的结合的问题。


第一种论证是达米特(M.Dummett)的语义学论证。达米特把实在论解释成这样一种学说:陈述可以独立于它是否被确证或证明(甚至独立于被确证或证明的可能性),而有或真或假确定的真值。或者说,陈述本然地具有独立于认知判定可能性的真值(在这个解释下,排中律应该非常自然地为实在论者所接受)。这个实在论定义,叫做“实在论的语义学解释(construal)”。当引入了语义学解释后,实际上,反实在论与实在论争议的背后,是对语言与世界关系的不同理解。

我们的语句所表达的,经常是我们不能直接认知地把握的,比如已经过去但当时末被核查的,\footnote{Dummett的例子是:考虑“最后一只恐龙是如何死的”。对于实在论的超验真值条件,不管我们认知上如何把握,关于它的命题总有真值或者反之;我们怎么通过根本无法找到的判据,去解释那种超验的真或者假?}或只是被计划的事情(我们所知道的,甚至直接认知所把握的,不能告诉我们语句所表达的事态是否obtain)。反实在论者声称,我们的实际语言使用,并没有给出实在论所说的独立于认知的真值的任何证据。事实上,这些语句如果有真值,它将依赖于我们给定的一些可断定性的条件,满足这些条件,我们就有理由断定其为真,这是科学中的典型做法。只是在我们的语言的实际使用中,在这些使用规则的执行中,语言表达式与被表达者的关系才建立起来。换句话说,在语言使用者中间公共地可共享的这些规则和被认定的条件,给了语言表达式以意义(客观性的根据不是实在于心灵的独立性,而是语言使用中能够公共地manifested的规则和条件;换句话说,放弃实在论的排中超验真值,而将意义从真值条件中解放出来,只一种保留与认知有联系的条件)。这里,我们用可断定性条件(Verificationists)代替独立于认知的真值条件(反实在论者认为,其背后的假设是“实在独立于语言与心灵”)。

来自Dummett和其他人的另一个论证是{\heiti 语言学习论证},它某种程度上是上面论证的连带物。其基本的形式蕴含这样几条:许多语句的真值是否被满足,无法通过直接的核查而确知(例如我们不能找到真值判据,而具有超验真值的那种句子);如果语句的意义来自于独立于心灵与语言的真值条件(根据实在论的语义学解释),这些语句是无法学习的;但是,这些语句事实上是可学习的,因此实在论是错的。

Putnam等的观点则是怀疑论的反实在论论证。其思路大体是,如果实在论是真的,则有心灵独立的实在,可是,心灵如何能把握独立于它的实在呢?这是一个传统的认识论难题,也是怀疑论提出的根据。这个认识论问题之无解,就开放了怀疑论为真的可能性:即我们所声称知道的全部都是错的是可能的。若认为怀疑论是不可接受的,我们就必须放弃实在论;阻止怀疑论的一个办法是放弃实在对心的独立性,因此,放弃实在论是阻止怀疑论的一个选项。进一步,可以有一个更强的,即指控实在论导致无意义性,或不一致性的论证。它的关键是假定实在论支持语义外在论:怀疑论者断言“我们是缸中之脑\footnote{Putnam的解释是:“一个人(可以假设是你自己)被邪恶科学家施行了手术,他的脑被从身体上切了下来,放进一个盛有维持脑存活营养液的缸中。脑的神经末梢连接在计算机上,这台计算机按照程序向脑传送信息,以使他保持一切完全正常的幻觉。对于他来说,似乎人、物体、天空还都存在,自身的运动、身体感觉都可以输入。这个脑还可以被输入或截取记忆(截取掉大脑手术的记忆,然后输入他可能经历的各种环境、日常生活)。他甚至可以被输入代码,‘感觉’到他自己正在这里阅读一段有趣而荒唐的文字。”有关这个假想的最基本的问题是:“你如何担保你自己不是在这种困境之中”?这是某种意义上怀疑论的一个例子。}(brains in a vat)”可能是真的(如果坚持实在论,我们必须承认怀疑论是可能真的),而如果语义外在论是真的(语义某种程度上是外在的因素所决定的,通常认为实在论者相信如此),怀疑论断言就不可能是真的,而是无意义的。因此,怀疑论与语义外在论(或许,实在论)的结合,将产生语义的不一致。

另一论证是“多对多”(Many-many)关系论证。实在论版本的表达主义假设了外在实体对语言和思想的独立性,于是,不可能保证两者之间的one-one关系。如果表达关系是意义的来源或决定者,那么,当我们不能确定一个语言或思想的载体对应于哪个外在对象时,意义就是不确定的。同时,当表达或映射关系可以分为“好的”或“坏的”,没有办法保证我们恰好选中了一个好的(如果被选择的对象独立于我们的认知);换言之,实在论的语义学解释和表达主义的组合将导致语义的不确定。因此,放弃实在论是一个选项\footnote{简单来说,世界和语言之间的关系既然根据实体的独立性不是一对一的,这将使得我们的语言找不到一个合理的外在对象去承载;而根据表达主义,语言就是要表达世界,但是语言和世界的对应关系甚至还不够清楚,这就产生了内生的矛盾。}。

进一步地,有一种“框架多元论”论证;框架多元论者相信,任何真理,包括关于存在的真理,都依赖于我们所采用的框架,因此,只是相对于框架才有意义,或者才是真的。如果框架多元论是真的,则实在论所要求的关于唯一实在的真理便没有意义,因此,实在论在多元论假定下是错的。进一步的麻烦是,可以有描述上完全等价(即被认为描述了同样的现象)的两个框架,但假设了不同的实体。问题是,这个现象说明什么?换言之,我们认为的同一事实,是可以有两种不同描述的,因此,语言与世界是否不可能有一一对应\footnote{举例来讲,三苹果问题,承认每个组合的独立性的普遍主义框架有7个对象,最小主义框架则只承认3个对象。同时就注意到,框架多元论前提下,这里没有一个实在的世界或者标准的框架独立于我们,使得我们通过那个世界的框架来具体考虑各种问题。}。

基于使真者理论,反实在论作为某种程度上支持使真者的论据被这一体系所论证,即使真者论证。在使真者视角下,存在两个难以解释的子类:第一个子类是关于某些X的断定,特别是否定存在断定不可能有实在使真者(例如否定事实),而第二个子类是本体论承诺和使真者承诺之间的差异导致的,亦即,奎因通过使用量词或指称手段而产生的本体语义承诺对象与使真者承诺之间的差距,或者虚构对象的反实在论;这里,具有指称虚构对象的那个语句可以真,有存在实在以外的使真机制\footnote{举例而言,涉及孙悟空的命题可以是真的, 因为吴承恩写的那本小说是其非虚构的使真者;但不必坚持本体论原则为孙悟空赋予实在性。}。

基于语用学的论证则指出,我们所有的那些关于抽象对象的命题只不过是假装为真,实则为假,当我们看起来在语义上断定了那些东西存在,但是事实上字面上命题是不真,只不过我们假装其真来解决一些问题罢了。换言之,当我们说A的时候,实际上我们说的是B,B是实在的。最后一种论证是指称的视角性论证(Putnam's perspectival reference),其基本论证结构是六条:真理依赖于指称,且由指称构成(至少部分);指称依赖于因果关系,并由因果关系(至少部分)构成;因果关系是视角的观点;从而,指称也是依赖于视角的观点;最后真理也成为一种依赖于视角的观点,亦即实在相对于某个观察者是视角性的\footnote{根本上的思路是,确定事件的原因取决于我们的视角,或者我们认为什么是重要和关键的;Sosa的例子是,引起森林火灾的原因可以有丢弃的香烟或氧气的存在两种解释,地球人的视角下原因将是前者,火星人则是后者。同时,Sosa指出,我们不应该从(5)中得出(6),因为对一个视角的语义依赖和对一个视角的形而上学依赖是有区别的。}。
\section{元形而上学}
所谓元形而上学(meta-metaphysics),它某种程度上是比形而上学更“高阶”的形而上学;形而上学研究那些因果、存在、世界的本质之类的对象,而元形而上学探讨的则是形而上学这门学科本身的性质,探讨形而上学学科本身的本质如何、能够解决的问题又如何;乃至,形而上学研究的问题是否有意义,我们有无必要针对形而上学的那些命题进行彻底的探讨。
\subsection{形而上学问题与回答的性质}
卡尔纳普(R.Carnap)首先对关于某个系统或者理论的问题做了内问题和外问题的划分。内问题是指,在一个系统内,在系统的确定的语义规则下被表达的问题。内存在问题是可能在系统内部判断为真的,它由这个系统的两个语言装置所限制:一个恰当的量词和量词域中的对象的一部分\footnote{这是一个相对较高阶的概念的外延,举物理学的例子来讲,语言装置蕴含“是一个物理对象”这一超越系统内部概念的外延,内问题例如全部科学问题:“以太是物理对象吗?”。后面说到的外问题则像是更高阶的“什么是物理对象”,它不在系统语言装置内部。}。内问题(当然,包括内存在问题)或者被分析的方法(依赖意义推导真值)、或者根据综合的方法(依赖经验的方法寻求解释)来回答。而外问题则是在系统之外针对系统的整体所提出的问题,它总之必须超越系统的范畴,或者关于这个系统对象的总体,或者是直接关于这个系统本身。其中有两种情形:其一是(Carnap的逻辑实证主义认为)有意义的问题,它研究实践性意义下一个系统是否恰当或者合理,是否合于某些实际的情形;另一种则被认为是无意义的,它将系统总体的问题当作可以用系统内部的那个理论方式能够回答的,而且是有超越语言装置而存在确定的真值的问题,举例而言,传统形而上学家研究的存在性问题(尽管存在就是其语言装置和理论所蕴含的总体性基础概念)。

针对这一内问题和外问题的区分,我们有两种解释。第一种是Quine对卡尔纳普区分的解释\footnote{Quine虽然解释了这一点,事实上并不支持卡尔纳普。其整体论拒斥做分析和综合的区别(而那是Carnap做内外区别所依赖的),形而上和科学的界限;他所坚持的是“信念之网”,指出若干信念之间是互相辩护的,从整体论视角上不能割裂,内外问题更是如此。},它指出内外的性质是子类(Subclass)和范畴问题的区别。范畴问题是考虑某一个量词域里所有的对象的那种问题,而子类问题则只是考虑域内对象的某一特定子类。Neo-Carnapian interpretation则把内外区别的判据认为是“可回答与不可回答”;然而需要解决的问题是,这能否保证形而上学问题(至少有时候)被有意义地提出并且回答(假如“可回答”只保障语言学上问题被语言规则允许,而不蕴含问题的真值问题等是否包含语言装置和确定真值)?

在这里,我们考虑分析与综合的区分的问题。回到Quine那里,我们来考虑他对卡尔纳普的分析和综合的阐释。在Quine著名的《经验论的两个教条》当中,他这样阐述卡尔纳普的两个教条:第一个教条是,存在着分析与综合陈述的确定的分界线,并且分析陈述的存在是全部必然性和先天性的唯一来源。其根据是,分析之所以成为分析根植于语言规则和相关的基本约定,并且这些规定并不对任何这个世界具体如何的问题做出解释\footnote{对于分析性这个概念,我们必须注意到它不能同先天必然的论述等价起来。Quine的分析陈述是“True in virtue of meaning or semantic  rules”(基于意义或者语义规则的),综合陈述则是“True at least partly in virtue of language-independent matters of fact”(至少部分基于独立于语言之外的情形),是一个语义学上的区分,即是否只能以意义而成真。这与先天和后天之间的那种认识论的区别是不同的。所以,分析判断的先天性来源于仅仅根据概念上的包含关系就得到陈述的真性,而必然性在于否定它必然某种程度上引发矛盾。这并不意味着只要是先天的必然判断就是分析的。Kripke和Kant只是在把先天和必然在范畴上分离的方式不同,但是他们都承认先天并非必然,而它们又不是分析的命题。};第二个教条则是所有科学中的理论陈述都可以通过系统的方式还原(Reduce to)成为观察陈述\footnote{观察语言亦称“基础语言”,包括观察名词和观察陈述(又称观察命题或观察语句)。观察名词是表示或指称可直接观察或测量的对象或过程以及这些对象的可观察属性和关系的语词。它们“必须是不涉及抽象的实体,而仅涉及可观察的对象或事件”。观察陈述是用观察名词作谓词的陈述。而理论语言是由理论名词和理论陈述所构成的语言;理论名词是表示不可直接观察的对象、事件或事件之间关系的名词。,像是场,中子等。}。

这里,被承认的(有意义的)论述是符合这样的条件的:只有两类论述,分析论述和综合论述,此外的论述都是不具备认知意义的。传统的形而上学被否认的理由是,它只是具备着陈述的表观形式,但是事实上不具备真正的陈述所应该具有的性质,是无意义的;进一步地,有很多仅仅具备陈述形式的“陈述”其实也并不具备被认可的认知意义。

Quine对分析性和综合性的区别做了批驳。卡尔纳普承认的分析性当中,第一类分析的真陈述是逻辑上为真的陈述,第二类则是通过同义替换而演变成为逻辑真的陈述。Quine在《两个教条》当中假定了第一类陈述的分析性(尽管他本人并不承认这一点),他的问题是:所谓“同义替换”得到的那个命题仍然成立分析性吗;或者,我们真的能够无问题地获得或者刻画同义性吗?事实上,同义性之所以成为分析性的核心内容,正是基于“通过同义代换,语言就能够演变成为逻辑真理”来考虑一般的陈述赋予分析性的,而Quine在他的文章当中批评的正是这一观点。

Quine的第一个论证是:不可能有分析性的非循环的定义。这分为两个子论证,一个是同义性如何定义的问题,另一个则是语义规则作为分析性定义是否可行的问题。他指出,试图列出分析真理并不可行,因为那也不能告诉我们它实际上是什么;利用别的什么去论证事实上还是用分析真理去说明分析真理,就陷入循环的情形了。第二个子论证是,用定义来刻画分析性仍然是循环的,因为定义要么基于已有的同义性的经验报告(例如一本辞典),要么就是简单的规定;二者都是以对同义性质的先验理解为基础,列表也不能一般地刻画分析性的形而上本质。

自然而然地,这论证遭受了旧有的逻辑实证主义的反驳。对第一个论证而言,P.F.Strawson的意见是,基本的概念不可能有那种严格的、非循环的由更基本的概念来保证的定义,我们必须不加推理地承认一些基本的事实和定义,而不能期望所有的分析性概念都必须有什么论证;对第二个论证而言,卡尔纳普的回应是:分析性并不像他所说的那样因为修改性就丧失了意义。事实上,分析性是系统内部的性质,分析陈述需要在某个给定的语言系统内部来定义;因而分析性并不是不可修改的,利用经验去修改分析陈述事实上只不过是换了一个新的系统而已。

Yablo对于区分的观点则更偏于语用学,他拒斥实在论而是虚构主义者。Yablo指出,这里的内外区分来自于语境的或者话语类型的区分;当我们真正想讨论一个形而上学问题,考虑一个命题的字面意义上的和实际上的真,我们就必须把系统内部的问题及其涉及的一般前设总体地讨论;而当我们仍然在那个系统内部去进行他所谓的“假装游戏”,即对本质上虚构的对象去语用地考虑的时候,我们研究的就是内问题,它蕴含于系统的内部\footnote{Yablo在这里的观点是,卡尔纳普的元形而上学事实上是某种紧缩主义。简单来说,对传统形而上学问题的研究并不是一种真正的研究,因为没有事实表明哪一种本体论是正确的。因此,形而上学的争论在理论上是空洞的。所有的那些本体论都没有解释意义,它只不过是概念真理的导出的轻实在论;形而上学意义在这个意义上是被拒斥的。}。最后一种观点认为内外之分是量词多域理论的一个特例,当量词域这个对象本身就具备着多元性的时候(或者本体论承诺的那个量词域是多元的),内问题和外问题本身就是不同的量词域限制下的对象。

\subsection{形而上学有意义的争论的探讨}
作为元形而上学的课题之一,形而上学家们之间的典型争论的意义是重要的课题。第一个争论的主要问题是语词之争\footnote{一种定义是:关于S的争论是语词之争当且仅当双方使用S表达了不同的命题P和Q, 并 且双方对于P或者Q之为真没有分歧。简单来说,这里考虑在哲学上事实问题和语言问题之间的区别。我们可以大致定义语词之争:争论的双方都同意相关的事实, 只是在如何使用语言去描述相关的事实上存 在不同,是为语词之争。一个例子是三苹果问题对象是3个还是7个。}(以及它的标准)。这里,一种观点相信,就语词问题而言,他们正在严肃而有意义地讨论一个有实际意义的问题。而否定者们指出,他们只是在用不同的形式去讨论同一个对象。所以,在他们的那种语言当中,关于对象存在的陈述都是真的,甚至在同样的那些可能世界为真,它们的图像完全是一样的,没有什么本质的区别(像是三对象和七对象问题那样),只不过是使用量词所约束的域不同罢了,框架性的不同并不是什么严肃的争论\footnote{一个形而上学的思想实验是:如果无神论者采用有神论的语言装置解释了他那无神论的观点,那么就说明了这里语言装置的区别并不影响实际的某种意义。}。

第二个关于争论的问题是探讨是否存在一个可以公度的关于形而上学问题的正确性标准。一般认为,如果这个标准存在的话,至少需要两个基础:第一,存在一个有唯一的作为事实存在的特殊的世界(像它实际所是的那样),作为两个说话者之间的公共世界;第二,有一个(或者相互之间本质不可区分的多个)最好地而且完善地刻画了这个独立存在世界的语言,它“精准的在那些节点的基本结构上刻画了那个世界”。T.Sider的观点是二者都存在(从而形而上学问题有可公度的标准),而其他哲学家则有一些否定其一乃至两者。

另一争论是形而上学的Deflationary(缩减性),我们之前在Yablo那里已经提到过这个问题。进一步的解释是,合法的形而上学问题应当很容易回答;尽管如此,这个回答并没有深刻的那种传统形而上学家所相信的意义。某种意义上,我们提出的形而上学的问题和它的回答只不过语词和概念所决定;进一步地,我们真正有争议的不在于世界本身,而是世界本身是怎么样被我们讨论和思考的。这之所以是元形而上学的理论,根植于其讨论的内容包含了形而上学的意义等。

进一步的争论是,形而上学存在问题(或者,其他基本问题)是绝对一般的吗\footnote{所谓绝对一般性,某种程度上含有最大的观点。以数学举例,考虑数学对象的存在性时,形而上学存在性如果满足绝对一般的要求,就应当某种程度上承认所有的数学对象都存在或者都不存在,只承认一些数学对象具有某种性质或存在的看法事实上并未保留那个“绝对一般性”,而是受限的或Local的。}?这个问题某种程度上同上面的可公度标准是统一的,涉及我们是否有一个一般的基本公共世界和刻画它的语言的讨论。关于形而上学研究的域的绝对一般性,我们往往有这样的看法:当我们被给出一个所谓的“存在性的反例”的时候,我们并不能认为这个抽象对象的形而上存在性的解答是开放的或者同某种形而上的阐释无关;根本上,不论它们存在与否,这些对象在承认绝对一般性时都必须落在我们的论域当中,不能拒斥任何对象;而形而上学需要给它的存在性一个解答。具体的问题有两个:关于形而上学的部分是,我们是否有一个一般地包括了所有对象的论域;而就存在性而言,需要知道的是一个绝对一般地包含了对象的域是否作为一个论域而存在。对形而上学问题的正面回答(像Sider那样)是:当然存在一个这样的绝对一般域,这就是所谓的绝对主义;非绝对主义者则承认,根本没有那样的绝对一般的域。需要注意的是,这个回答似乎仍然使用了绝对主义的范式;这也是绝对主义者所乐于承认的。

针对绝对主义的反驳有这样的几条。首先是来自Dummett的扩展性的讨论:存在一些能够无限拓展的概念,这些概念的外延总可以引发新的外延;而对于具有不确定的外延的例子,我们总是能够找到“新的外延”。Dummett的例子是最大集合这个概念的不存在性\footnote{具体方法是利用罗素悖论:对于任意一个集合A,A要么是自身的元素,即$A\in A$;A要么不是自身的元素,即$A\notin A$。根据康托尔集合论的概括原则,可将所有不是自身元素的集合构成一个集合$S_1$,即$S_1=\{x:x\notin x\}$,这就不断的使得集合扩展。};既然外延的扩张没有限制,那么最大的论域自然也不存在。这一反驳的问题是,它究竟是语言学限制还是形而上学有限性的限制;或许我们的语言确实难以表达出那个最大的绝对方面,但是不能因此断定它形而上地不存在。

第二种反驳来自框架相对主义、约定主义和X多元论等支持非单一的观点。这些理论的共同特点是,认为论域这个对象不应该是相对单一的,而是可选择的或者多元的。框架相对主义事实上可以参考之前的框架多元论和语义敏感词那里的相对主义去理解,它思考哪些东西是依赖于框架的;对于经典的“3 Objects or 7 Objects”的组合性问题,我们之前在框架多元论那里已经说明了它根本上是普遍主义和最小主义框架的区别;这里则更一般地谈论,所谓绝对一般性之所以不能够被接受,是因为我们考虑的对象事实上依赖于框架、约定或者某种所关于的主题的多元性和相对性(而不是基于一个简单的绝对唯一论域);当然这个提法也面临着类似的问题,即语言学限制还是形而上存在的限制的问题。Sider的看法是,依照他在形而上学问题公度那里的讨论,应该存在一种“基本语言”\footnote{其定义是:若L是“基本语言”(Joint Carving Launguage),那么我们依靠这种语言就可以精细地完善地复原世界的那种结构。这事实上仍然基于“世界是有真正的结构作为架构的基础的,有一个语言能够最好的刻画”这两个前提。}和“基本框架”去刻画前面所说的那种世界;换言之,反对多元论中不存在一个绝对一般的特殊对象的那种看法。

第三种反驳基于Quine原始翻译理论的语义不确定性。既然确定的意义是不可能得到的,那么也就是说,这里有不同的(但是却同样合理的)方案去解释问题;从而,一个论断究竟支持绝对主义还是非绝对主义有可能其实是不确定的,它甚至可以同两者都相容,因为论断本身基于的语义学基础就是一个不确定的对象,那么自然就不可能有真正的支持绝对主义的论证。更进一步地,我们刻画绝对一般性的过程当中所需要用到的“对象”这样的类词本身就没有明确的外延,更不存在什么确定的刻画去给出其外延和蕴含,自然对于绝对一般性的论证本身也不存在什么确定的刻画。
\subsection{*科学与形而上学的分界}
Quine作为整体论者否定这里存在着根本的不同,他认为经验科学同形而上学根本没有什么本质的区别,是融贯的;Lowe则指出,区别是基于主题和方法的。对主题而言,形而上学是关于可能性的探讨(而非自然科学的实证);方法上则认为形而上学是先天的认知,科学则需要后天的必然性。

最后一种观点来自于经验的自然主义者Van Fraassen,它指出形而上学本身的无意义性。作为元形而上学讨论的最后一部分,这一观点至少得到了我们很多人的认可,让我们最后用他的三条观点来结束关于形而上学的讨论:

{\kaishu 第一,形而上学是远离经验从而无用的。}

{\kaishu 第二,它根本没有一个确定的回答问题的范式。}

{\kaishu 第三,形而上学没有什么基本的可理解的概念。}
\end{document}
